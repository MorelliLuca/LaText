\section{Il modello epidemiologico SIR}

Vogliamo studiare ora la diffusione di una pandemia, per farlo analizzeremo il modello più semplice possibile ossia il modello \textbf{SIR}. Consideriamo una popolazione di \textbf{N} individui fissi, assunzione che possiamo fare se pensiamo di studiare la popolazione per un tempo sufficientemente corto per cui le persone decedute sono approssimativamente compensate da quelle nate. \\
La popolazione è quindi suddivisa in tre sottosistemi \textbf{S}, i suscettibili, \textbf{I}, gli infetti, e \textbf{R}, i recovered ossia i guariti. In questo caso l'assunzione che N sia costante ci consente di introdurre un sorta di \textbf{legge di conservazione}:
\begin{equation}
	S+I+R=N
	\label{continuità}
\end{equation} 
Inoltre dobbiamo definire delle leggi che consentano modellizzare i link tra i tre sottosistemi, queste ovviamente devono rispettare la (\ref{continuità}) e quindi nella loro forma più semplice appaiono come:
\begin{equation}
	\begin{cases}
		\dot{S}=-\Phi_{S\rightarrow I}(t)\\
		\dot{I}=\Phi_{S\rightarrow I}(t)-\Phi_{I\rightarrow R}(t)\\
		\dot{R}=\Phi_{I\rightarrow R}(t)\\
	\end{cases}
\label{flussi}
\end{equation}
Così facendo il problema si sposta sulla necessità di trovare le 3 funzioni dei flussi tra i sottosistemi:
\begin{itemize}
	\item $\Phi_{S\rightarrow I}(t)$ è facilmente definito infatti è chiaro che se ogni individuo può infettarsi allora il flusso sarà dato dal numero di suscettibili per la probabilità che incontrino un infetto, stimata con $\frac{I}{N}$, e per un fattore $m(t)$ che misura la socialità, per cui abbiamo :
	\begin{equation}
		\Phi_{S\rightarrow I}(t)=\beta S\frac{I}{N}m(t)
	\end{equation}
    \item $\Phi_{I\rightarrow R}(t)$ è più complesso da ricavare, infatti deve tenere conto del ritardo dovuto al tempo di guarigione, per farlo consideriamo la seguente espressione:
    \begin{equation}
    	\Phi_{I\rightarrow R}(t)=\int_{0}^{\infty}\Gamma_I(s) \Phi_{S\rightarrow I}(t-s)ds
    \end{equation}
     Questa ci consente di stimare quante persone che si sono infettate al tempo $t-s< t$ guariscono al tempo $t$, moltiplicandole per la probabilità $\Gamma_I(s)$ di guarire dopo un tempo $s$ che sono infette. 
\end{itemize}
\vspace{5pt}

Solitamente la probabilità di guarire è data da una particolare funzione $\Gamma_I(s)=cs^ae^{-bt}$, che chiaramente tende a zero sia per $s$ piccolo che grande, fornendo quindi un punto in qui la probabilità di guarire è massima.\\ Risulta però molto complicato utilizzare questo integrale come definizione del flusso per cui introdurremo una serie di approssimazioni e semplificazioni: in primo luogo considereremo un valore medio della probabilità ($\gamma$) e lo assumeremo costante durante l'integrazione, in secondo luogo supponiamo che $\Phi_{S\rightarrow I}(t)>>\Phi_{I\rightarrow R}(t)$, cosa effettivamente vera durante le prime fasi dell'epidemia, in questo modo possiamo approssimare $\Phi_{S\rightarrow I}(t)\backsimeq\dot{I}$ che se integrato restituisce $I(t)$, poichè per tempi molto precedenti la pandemia non è ancora incominciata e quindi risulta nullo il numero di infetti. Otteniamo quindi dalla (\ref{flussi}) un modello più semplice ma ancora utile:
\begin{equation}
	\begin{cases}
		\dot{S}=-m(t)\beta S\frac{I}{N}\\
		\dot{I}=m(t)\beta S\frac{I}{N}-\gamma I\\
		\dot{R}=\gamma I\\
	\end{cases}
	\label{SIR}
\end{equation}

Prima di iniziare a studiare il modello procediamo a riscalare le variabili in maniera tale da normalizzare la popolazione, imponiamo infatti che $N=1$ e studieremo le variazioni percentuali da qui in poi.
\subsection{Studio del modello}

Per prima cosa osserviamo che se $I=0$ chiaramente il sistema è in equilibrio, per cui studieremo quali condizioni consentono ad una epidemia di svilupparsi e come questo sviluppo avviene.\\

Perchè l'epidemia si diffonda dobbiamo introdurre almeno un infetto, in questo caso però piccole perturbazioni di $I$ lasciano pressochè invariato $S=1$, per cui abbiamo come condizione per lo svilupparsi dell'epidemia:
\begin{equation}
	\dot{I}=(-m_0\beta-\gamma )I \Rightarrow m_0\beta>\gamma
\end{equation} 
ossia che i parametri che influenzano la diffusione devono "\textit{surclassare}" quelli relativi alla guarigione.\\

Posso a questo punto proseguire nello studio osservando che con una semplice sostituzione di variabili è possibile ridurre il modello ad un sistema \textbf{Hamiltoniano} di cui ben conosciamo le caratteristiche:
\begin{equation*}
	\begin{gathered}
		\frac{\dot{S}}{S}=\frac{d}{dt} \log S =-m(t)\beta I\qquad\log S=u\qquad \dot{u}=-m(t)\beta e^{v}=-\frac{\partial\mathcal{H}}{\partial v}\\
			\qquad\qquad\Rightarrow\qquad\\
			\frac{\dot{I}}{I}=\frac{d}{dt}  \log S =m(t)\beta S-\gamma\qquad\log =v\qquad\dot{v}=m(t)\beta e^{u}-\gamma=\frac{\partial\mathcal{H}}{\partial u}\\
	\end{gathered}
\end{equation*}
Così facendo troviamo un integrale primo del moto che ci consentirà di fare alcune osservazioni utili, da qui in poi però dobbiamo assumere che $m(t)$ sia costante:
\begin{equation}
	\mathcal{H}(I,S)=m\beta(I+S)-\gamma\log S=m\beta \label{hamilton}
\end{equation}
Così facendo, conoscendo le condizioni iniziali del sistema e quindi il valore dell'integrale primo, possiamo determinare in quanti punti vale $I=0$ : infatti un primo punto che non ha infetti è il punto iniziale e se vi fosse un secondo punto in tal caso avremo la fine della pandemia. Studiando la  (\ref{hamilton}) si scopre facilmente che la condizione perchè l'epidemia possa avere luogo sono anche sufficienti per la sua fine; in questo caso bisogna risolvere un'equazione non algebrica per cui è necessario uno studio di funzione molto banale che però omettiamo (perchè non ho voglia ed è una cazzata).\\

Vogliamo a questo punto prevedere quando si manifesterà il picco di infetti e comprenderne l'entità. Ricordando, dal cambio variabili di prima, che $\frac{\dot{S}}{S}=\frac{d}{dt} \log S =-m\beta I$, possiamo ridurre la (\ref{hamilton}) alla seguente forma:
\begin{equation*}
	-\frac{\dot{S}}{S} + m\beta S-\gamma\log S=m\beta
\end{equation*}
a questo punto approssimiamo in serie di Taylor $\log S\simeq1-S$ ottenendo così un'equazione differenziale che abbiamo già incontrato, ossia l'equazione logistica, che però è valida solo per $S\sim 1$, ovvero solo se l'epidemia è agli inizi o ad essere infetti non è una percentuale troppo grossa della popolazione (se $S=0,6$ l'errore $\simeq 15\%$).
\begin{equation}
	\dot{S}=(\beta m-\gamma)(1-S)S
\end{equation} 

Infine osserviamo che se $\frac{\dot{S}}{S}=-m\beta I$ e $S\simeq 1$ allora possiamo stimare a grandi linee il punto di massimo di $I$ come il punto di flesso di $S$, infatti:
\begin{equation}
	\dot{I}=\frac{\ddot{S}S-\dot{S}^2}{S^2}=\frac{\ddot{S}}{S}-I^2
\end{equation}
e questa espressione tende a $0$ se $I<<S$ e $\ddot{S}=0$. Questa stima in realtà è del tutto generale e infatti $\frac{\dot{S}}{S}=-m\beta I$ è immediatamente ricavabile dalla (\ref{SIR}).