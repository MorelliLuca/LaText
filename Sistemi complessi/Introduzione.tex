\section{Introduzione}

Fino ad ora la fisica è riuscita a descrivere moltissimi fenomeni con l'uso di due approcci diametralmente opposti: da un lato riusciamo a descrivere con estrema precisione sistemi estremamente semplici, come avviene in meccanica, dall'altro sistemi molto complessi con un'approccio statistico, come avviene in termodinamica e in meccanica statistica. \\

Siamo infatti in grado di descrivere perfettamente l'orbita della Terra attorno al Sole fin tanto consideriamo un sistema a due corpi ma nel momento in cui aggiungiamo per esempio l'interazione con la Luna incappiamo nel \textbf{problema dei tre corpi} che non è risolvibile analiticamente ma può essere risolto con metodi di calcolo numerico al calcolatore.\\

Proseguendo aumentando il numero di corpi in questione non siamo nemmeno più in grado di sottoporre il problema alla risoluzione del calcolatore, come avviene nel caso delle molecole di un gas. Siamo però capaci di descrivere statisticamente un gas tramite l'analisi \textbf{all'equilibrio} della sua pressione, della sua temperatura e del suo volume. 

Proprio nella zona di grigio tra queste due casistiche entra in gioco la \textbf{fisica dei sistemi complessi}.
\subsection{Cos'è la complessità}

Come già anticipato parliamo di \textbf{complessità} quando vogliamo descrivere la dinamica di un sistema che presenta:

\begin{itemize}
	\item Molti gradi di libertà
	\item Interazione non trascurabile tra le parti
	\item Troppe condizioni iniziali da conoscere 
\end{itemize}

Per far questo costruiamo un modello che deve essere messo alla prova con la realtà sperimentale e che ci consenta di studiarla matematicamente.

Un esempio di questi si sistemi da modellizzare può essere l'evoluzione di una popolazione o di una pandemia.
