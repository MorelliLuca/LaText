\section{Introduction}
General relativity, based upon the \textbf{principle of general relativity}, which states that the laws of physics are the same for all observers, describe gravitational interactions through the geometry of the universe. 
\\ The universe is modelled via a manifold, which charts represent different reference frames, and gravity is described by the metric tensor. Free-falling observers, that are not subject to forces like the electromagnetic one, following geodesics on the manifold will deviate their path, from the "euclidean" straight line, due to the metric of the manifold.\\ The metric tensor is determined by the \textbf{Einstein field equation} $$R_{\mu\nu}-\frac{1}{2}Rg_{\mu\nu}=8\pi GT_{\mu\nu},$$
which relates derivatives of the metric, contained in the components of $R_{\mu\nu}$ (Ricci tensor) and in $R$ (Ricci scalar), to the energy-momentum tensor $T_{\mu\nu}$, describing the density of energy and momentum of the matter (but also radiation and other things) contained by the universe.\\

This deep connection between the geometry of spacetime and the physics of the universe, allows us to study the nature of it, what kind of manifold really it is, its curvature, its time evolution, knowing its content. This role is played by the \textbf{Friedmann equations}, which determine the cosmological parameters describing the universe. These will characterize the \textbf{Robertson Walker metric}, the metric that describe an isotropic and homogeneous universe. \\

In this short essay, we will first derive, from the \textbf{cosmological principle}, the Robertson Walker metric. We will describe the main features of it and the possible geometries that it can represent. Then, we will derive the Friedmann equation and describe the main cosmological parameters that are contained in it.