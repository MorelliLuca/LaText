\section{The geometry of the universe}
Modern cosmology is based upon two basics principles:
\begin{itemize}
    \item \textbf{the Copernican Principle}, or that there not exist a preferred observer in the universe;
    \item \textbf{the Cosmological Principle}, or that the universe is homogeneous and isotropic.
\end{itemize}
These principles may not seem to be consistent with physical reality, clearly the core of a star is very different from the empty space or even from the intern of planets, but in order to describe the whole universe we need to make some assumption that are valid on the largest scales. Observations, for example of the distribution of galaxies or of the cosmic microwave background radiation, show that at large scales, on average, the universe looks the same in all directions. From the Copernican Principle we than get that all observer should see an isotropic universe, thus we can claim that all points of the universe should also look the same. Again we should stress that these are just assumptions that, at some large scale, we think that can become adequate to approximate the description of spacetime.\\

We now have to translate the proprieties of isotropy and homogeneity to the language of different geometry and manifolds.\\ Notice that the two above principles refer only to the universe, or better, to space at a fixed time, therefore it is space which is really isotropic and homogeneous, while time has no particular symmetries.
Thus, space is \textbf{maximally symmetric}, or simply that it possesses the maximum number of killing vectors. In fact, homogeneity guarantees 3 killing vectors, corresponding to the 3 possible space translations, while isotropy guarantees other 3 killing vectors, the 3 rotations around a point, and the maximum number of independent killing vectors for a $3D$ manifold is indeed 6.
\subsection{The Robertson-Walker metric}
We will now construct a set of charts that are the more convenient to describe the assumed geometry.\\
First, consider a hypersurface $\Sigma$ (a volume in this case), which is a slice of the spacetime manifold, corresponding to space (the universe) at a fixed time. On this hypersurface we chose one chart with coordinates $x^\mu=(0,x^1,x^2,x^3)$. At each point $P\in\Sigma$ we pick a vector $\vec n$ that is orthogonal to $\Sigma$ (it should be orthogonal to each vector of the tangent space, in $P$, of the submanifold defined by $\Sigma$ and our chart) such that those are normalized to $-1$, since they must be time-like. In each point $P$ we can now build a unique geodesic, for which $\vec n$ is the tangent vector in $P$, from the following Chaucy problem
\begin{equation}\label{Chaucy problem}
    \begin{cases}
        \big(\nabla_{\vec n}\vec n\big)^\mu=\frac{d^2x^\mu}{dt^2}+\Gamma_{\nu\lambda}^{\mu}\frac{dx^\nu}{dt}\frac{dx^\lambda}{dt}=0,\\\frac{dx^\mu}{dt}\big|_{P}=n^\mu\big|_P,\\x^\mu(0)=x^\mu\big|_P.
    \end{cases}
\end{equation}
If $t$ is the parameter of each geodesic, we can extend our initial chart in a neighborhood of $\Sigma$ assigning to each point $Q$ the coordinates $x^\mu=(t,x^1,x^2x^3)$, where $t$ is the value of the parameter of one geodesic, that we have constructed, passing through $Q$, and $(x^1,x^2x^3)$ are the coordinates of the point $P$, from which the geodesic starts. These coordinates will eventually fail once some geodesics, from our construction, will meet and intersect.\\

We now want to describe the metric of our spacetime manifold using one of these charts. To do so we will take the chart induced basis of each tangent space $(\partial_t,\partial_1,\partial_2,\partial_3)$ and then label them:
\begin{equation}
    \partial_t=\vec n,\qquad \partial_i= \vec Y_{(i)},
\end{equation}
where the components of $\partial_t$ are really the ones of the normal vectors we have defined since they are parallel transported along the geodesics and defined by \ref{Chaucy problem}.\\
Using this basis the first component of the metric reads, by our initial construction,
\begin{equation}
    g_{tt}=g(\partial_t,\partial_t)=n^\mu n_\mu=-1.\label{gen gtt}
\end{equation} 
On $\Sigma$, from our construction hypothesis $\vec{n}\perp\Sigma $, the time-spacial mixed components read 
\begin{equation}
    g_{ti}=g(\partial_t,\partial_i)=n_\mu Y^\mu_{(i)}=0.\label{gen gti}
\end{equation} 
We can prove that this holds also outside $\Sigma$ by evaluating its covariant derivative along one of the geodesics we constructed
\begin{align*}
    n^\nu\nabla_\nu(n_\mu Y_{(i)}^\mu)&=n^\nu n_\mu \nabla_\nu( Y_{(i)}^\mu)+Y_{(i)}^\mu n^\nu  \nabla_\nu(n_\mu )\\
    &=n^\nu n_\mu \nabla_\nu( Y_{(i)}^\mu)\\
    &=Y^\nu_{(i)}n_\mu\nabla_\nu(n^\mu)\\
    &=\frac{1}{2}\big(Y^\nu_{(i)}n_\mu\nabla_\nu(n^\mu)+Y^\nu_{(i)}n_\mu\nabla_\nu(n^\mu)\big)\\
    &=\frac{1}{2}\big(Y^\nu_{(i)}n_\mu\nabla_\nu(n^\mu)+Y^\nu_{(i)}n_\mu\nabla_\nu(g^{\mu\lambda}n_{\lambda})\big)\\
    &=\frac{1}{2}\big(Y^\nu_{(i)}n_\mu\nabla_\nu(n^\mu)+Y^\nu_{(i)}n_\mu\nabla_\nu(g^{\mu\lambda})n_{\lambda}+Y^\nu_{(i)}n_\mu g^{\mu\lambda}\nabla_\nu(n_{\lambda})\big)\\
    &=\frac{1}{2}\big(Y^\nu_{(i)}n_\mu\nabla_\nu(n^\mu)+Y^\nu_{(i)}n^\lambda\nabla_\nu(n_{\lambda})\big)\\
    &=\frac{1}{2}Y^\nu_{(i)}\nabla_\nu(n^\lambda n_\lambda)=0,
\end{align*}
in which we used (in order): the geodesic equation $n^\nu  \nabla_\nu(n_\mu )=0$, that coordinates vectors commute, so that $[\vec n,\vec Y_{(i)}]^\mu=n^\nu \nabla_\nu( Y_{(i)}^\mu)-Y^\nu_{(i)}\nabla_\nu(n^\mu)=0$, the metric connection condition $\nabla g=0$, and last that, being $n^\mu n_\mu=-1$, its derivative vanishes.\\
Summing up all the above results, we can write the metric, from \eqref{gen gtt} and \eqref{gen gti}, as
\begin{equation*}
    ds^2=-dt^2+g_{ij}dx^i dx^j.
\end{equation*}
In this expression the absence of the mixed terms $dt\ dx^i$ reflects that there exist a family of hypersurfaces, defined by $t=$const, that are all orthogonal to the vector field $\vec n$. These represent the evolved universe at different times.\\
The spacial components of the metric now depend on all the coordinates of the chart we have introduced. If we consider how time evolution could affect the spacial terms we can deduce that all the components $g_ij$ should scale in the same way, otherwise we could have different scaling in different directions, which is against the idea that space is isotropic. We will write explicitly the time dependence as
\begin{equation*}
    ds^2=-dt^2+a^2(t)g_{ij}dx^i dx^j.
\end{equation*} 
Let's now take into account that each space hypersurface is a maximally symmetric submanifold.