\section{Teoremi}\index{Teoremi}

In questa sezione si mostrano esempi di teoremi.

\subsection{Più equazioni}\index{Teoremi!Più equazioni}
Questo teorema consiste di più equazioni.

\begin{theorem}[Nome del teorema]
In $E=\mathbb{R}^n$ tutte le norme sono equivalenti. In oltre valgono le proprietà:
\begin{align}
& \big| ||\mathbf{x}|| - ||\mathbf{y}|| \big|\leq || \mathbf{x}- \mathbf{y}||\\
&  ||\sum_{i=1}^n\mathbf{x}_i||\leq \sum_{i=1}^n||\mathbf{x}_i||\quad\text{$n$ intero finito}
\end{align}
\end{theorem}

\subsection{Teorema a singola riga}\index{Teoremi!Teorema a singola riga}
Questo è un teorema che sta su un sola riga.

\begin{theorem}
Un insieme $\mathcal{D}(G)$ è denso in $L^2(G)$, $|\cdot|_0$. 
\end{theorem}

\subsection{Dimostrazione}

Questo è un esempio di dimostrazione.
\begin{proof}
    Siano $a,b\in\mathbb{R}$, allora per definizione $|a|\geq a$ e $|b|\geq b$, da questo segue che la somma $a+b\leq|a|+|b|$.\\
    Si consideri ora $-a$ e $-b$, chiaramente i loro valori assoluti sono pari rispettivamente a quelli di $a$ e $b$, segue quindi che $-(a+b)\leq|a|+|b|$. Queste due considerazioni implicano che $|a+b|\leq|a|+|b|$.
\end{proof}

%------------------------------------------------

\section{Definizioni}\index{Definizioni}

Questo è un esempio di definizione. Una definizione può essere matematica o concettutale.

\begin{definition}[Nome definizione]
Dato uno spazio vettoriale $E$, una norma in $E$ è un'applicazione, indicata con $||\cdot||$, $E$ in $\mathbb{R}^+=[0,+\infty[$ tale che:
\begin{align}
& ||\mathbf{x}||=0\ \Rightarrow\ \mathbf{x}=\mathbf{0}\\
& ||\lambda \mathbf{x}||=|\lambda|\cdot ||\mathbf{x}||\\
& ||\mathbf{x}+\mathbf{y}||\leq ||\mathbf{x}||+||\mathbf{y}||
\end{align}
\end{definition}

%------------------------------------------------

\section{Notazioni}\index{Notazioni}

\begin{notation}
Dato un sottoinsieme aperto $G$ di $\mathbb{R}^n$, l'insieme di funzioni $\varphi$ sono:
\begin{enumerate}
\item Limitate su $G$;
\item di classe $C^\infty$;
\end{enumerate}
uno spazio vettoriale è indicato con $\mathcal{D}(G)$. 
\end{notation}

%------------------------------------------------

\section{Osservazioni}\index{Osservazioni}

Questo è un esempio di osservazione.

\begin{remark}
Quanto illustrato precedentemente rappresenta uno fondamenti della matematica. Gli spazi vettoriali sono costruiti su campi $\mathbb{K}=\mathbb{R}$, però, è facile estendere tutte le proprietà a $\mathbb{K}=\mathbb{C}$.
\end{remark}

%------------------------------------------------

\section{Corollari}\index{Corollari}

Questo è un esempio di corollario.

\begin{corollary}[Nome corollario]
Quanto illustrato precedentemente rappresenta uno fondamenti della matematica. Gli spazi vettoriali sono costruiti su campi $\mathbb{K}=\mathbb{R}$, però, è facile estendere tutte le proprietà a $\mathbb{K}=\mathbb{C}$.
\end{corollary}

%------------------------------------------------

\section{Proposizioni}\index{Proposizioni}

Questo è un esempio di proposizione.

\subsection{Più equazioni}\index{Proposizioni!Più equazioni}

\begin{proposition}[Nome proposizione]
Si hanno le seguenti propietà:
\begin{align}
& \big| ||\mathbf{x}|| - ||\mathbf{y}|| \big|\leq || \mathbf{x}- \mathbf{y}||\\
&  ||\sum_{i=1}^n\mathbf{x}_i||\leq \sum_{i=1}^n||\mathbf{x}_i||\quad\text{where $n$ is a finite integer}
\end{align}
\end{proposition}

\subsection{Teorema a singola riga}\index{Proposizioni!Teorema a singola riga}

\begin{proposition} 
Sia $f,g\in L^2(G)$; Se $\forall \varphi\in\mathcal{D}(G)$, $(f,\varphi)_0=(g,\varphi)_0$ allora $f = g$. 
\end{proposition}

%------------------------------------------------

\section{Esempi}\index{Esempi}

Questo è un esempio di esempio.

\subsection{Equazioni e Testo}\index{Esempi!Equazioni e Testo}

\begin{example}
Sia $G=\{x\in\mathbb{R}^2:|x|<3\}$ e indicato con: $x^0=(1,1)$; considerando la funzione:
\begin{equation}
f(x)=\left\{\begin{aligned} & \mathrm{e}^{|x|} & & \text{si $|x-x^0|\leq 1/2$}\\
& 0 & & \text{si $|x-x^0|> 1/2$}\end{aligned}\right.
\end{equation}
La funzione $f$ ha supporto compatto, si può prendere $A=\{x\in\mathbb{R}^2:|x-x^0|\leq 1/2+\epsilon\}$ for all $\epsilon\in\intoo{0}{5/2-\sqrt{2}}$.
\end{example}

\subsection{Paragrafo di Testo}\index{Esempi!Paragrafo di Testo}

\begin{example}[Nome esempio]
\lipsum[2]
\end{example}

%------------------------------------------------

\section{Esercizi}\index{Esercizi}

Questo è un esempio di esercizio.

\begin{exercise}
Gli esercizi sono utili per gli studenti.
\end{exercise}

%------------------------------------------------

\section{Problemi}\index{Problemi}

\begin{problem}
Se una mela cade da un albero di $10m$ con velocità finale $22m/s$ qual è la massa della Via Lattea?
\end{problem}

%------------------------------------------------

\section{Vocabolario}\index{Vocabolario}

Qui è possibile dare ulteriori informazioni per parole specifiche utilizzate.

\begin{vocabulary}[Parola]
Definizione di Parola.
\end{vocabulary}