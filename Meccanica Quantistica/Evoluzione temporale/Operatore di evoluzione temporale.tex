\section{Equazione di Schrödinger}
Come si è visto, uno stato quantomeccanico consente di ottenere gli osservabili del sistema a cui appartiene attraverso l'uso di operatori autoaggiunti. Inoltre si è sempre fatto riferimento ad uno stato definito in un istante e di cui si vogliono misurare alcuni osservabili. Si vuole ora descrivere l'evoluzione temporale di tale stato.\\

Sia $\ket{\psi_0}$ lo stato del sistema ad un istante iniziale $t=0$, si indicherà con $\ket{\psi(t)}$ il suo stato ad un istante successivo $t$. L'evidenza sperimentale mostra che gli stati $\ket{\psi(t)}$ sono soluzione della nota \emph{equazione di Schrödinger}:
\begin{equation}
    i\hslash\frac{d\ket{\psi(t)}}{dt}=\hat H(t)\ket{\psi(t)},\label{eqSchr}
\end{equation}
dove $\hat H(t)$ è l'operatore hamiltonano, ovvero l'operatore associato all'energia del sistema in un istante $t$. Imponendo $\ket{\psi(0)}=\ket{\psi_0}$ si ottiene il problema di Cauchy che ha come soluzione l'evoluzione temporale dello stato.
\section{L'operatore di evoluzione temporale}
Si osservi che l'equazione di Schrödinger \eqref{eqSchr} è un'equazione lineare, questo ha due dirette conseguenze:
\begin{itemize}
    \item se così non fosse non varrebbe il Principio di sovrapposizione,
    \item si può esprimere ogni stato come combinazione lineare di altri stati precedenti (quindi esiste un operatore lineare che fa questa operazione).
\end{itemize}
Si desidera costruire un operatore che descriva l'evoluzione del sistema associando ad uno stato iniziale il suo evoluto.
\begin{definition}
    Si definisce $\hat U(t,0)$ \emph{operatore di evoluzione temporale} se dato uno stato $\ket{\psi(t)}$ vale: $\ket{\psi(t)}=\hat U(t,s)\ket{\psi(s)}$.
\end{definition}
Per determinare la forma di questo operatore si può utilizzare l'equazione di Schrödinger \eqref{eqSchr} (infatti è questa che determina l'evoluzione dello stato):
\begin{flalign*}
    i\hslash\frac{d}{dt}\ket{\psi(t)}&=i\hslash\frac{\partial\hat U(t,s)}{\partial t}\ket{\psi(s)}=\hat H(t)\ket{\psi(t)}=\hat H(t)\hat U(t,s)\ket{\psi(s)}\\
    \Rightarrow&\boxed{i\hslash\frac{\partial\hat U(t,s)}{\partial t}=\hat H(t)\hat U(t,s)}\ .\\
    0=i\hslash\frac{d}{ds}\ket{\psi(t)}&=i\hslash\frac{\partial\hat U(t,s)}{\partial s}\ket{\psi(s)}+i\hslash\hat U(t,s)\frac{d}{ds}\ket{\psi(s)}\\&=i\hslash\frac{\partial\hat U(t,s)}{\partial s}\ket{\psi(s)}+\hat U(t,s)\hat H(s)\ket{\psi(s)}\\
    \Rightarrow&\boxed{i\hslash\frac{\partial\hat U(t,s)}{\partial s}=-\hat U(t,s)\hat H(s)}\ .
\end{flalign*}
Queste due equazioni differenziali costituiscono un problema di Cauchy se si aggiunge la banale condizione $\hat U(t,t)=\hat I$.\\Queste due equazioni garantiscono, all'operatore di evoluzione, una serie di proprietà.
\begin{proposition}
    Per l'operatore di evoluzione temporale valgono le seguenti proprietà:
    \begin{itemize}
        \item è unitario $\hat U^*(t,s)=\hat U(t,s)^{-1}$,
        \item $\hat U(t,u)\hat U(u,s)=\hat U(t,s)$,
        \item $\hat U(t,s)=\hat U(s,t)^{-1}$.
    \end{itemize}
\end{proposition}
\begin{proof}
    La prima relazione è dimostrabile utilizzando la prima delle due equazioni differenziali e osservando che facendone l'aggiunto di ambo i membri si ottiene:
    \begin{equation*}
        \hslash\frac{\partial\hat U^*(t,s)}{\partial t}=-\hat U^*(t,s)\hat H(t).
    \end{equation*}
    Si ha così che il prodotto $\hat U^*(t,s)\hat U(t,s)$ non dipende da $t$, infatti:
    \begin{flalign*}
        i\hslash \frac{\partial }{\partial t}[\hat U^*(t,s)\hat U(t,s)]&=i\hslash \frac{\partial \hat U^*(t,s)}{\partial t}\hat U(t,s)+i\hslash \hat U^*(t,s)\frac{\partial\hat U(t,s) }{\partial t}\\&=-\hat U^*(t,s)\hat H(t)\hat U(t,s)+\hat U^*(t,s)\hat H(t)\hat U(t,s)=0.
    \end{flalign*}
    Utilizzando ora la condizione iniziale $\hat U(t,t)=\hat I$ si ha:
    \begin{equation*}
        \hat U^*(t,s)\hat U(t,s)=\hat U^*(s,s)\hat U(s,s)=\hat I^*\hat I=\hat I
    \end{equation*}
    che implica la prima proprietà.\\
    La seconda proprietà è dimostrata dalla seconda equazione differenziale:
    \begin{flalign*}
        i\hslash \frac{\partial }{\partial u}[\hat U(t,u)\hat U(u,s)]&=i\hslash \frac{\partial \hat U(t,u)}{\partial u}\hat U(u,s)+i\hslash \hat U(t,u)\frac{\partial\hat U(u,s) }{\partial u}\\&=-\hat U(t,u)\hat H(u)\hat U(u,s)+\hat U(t,u)\hat H(u)\hat U(u,s)=0.
    \end{flalign*}
    Ne consegue che $\hat U(t,u)\hat U(u,s)$ non dipende da $u$, per cui:
    \begin{equation*}
        \hat U(t,u)\hat U(u,s)=\hat U(t,s)\hat U(s,s)=\hat U(t,s).
    \end{equation*}
    Infine, se nel primo membro di quest'ultima equazione si sostituisce $t\rightarrow s$ e $u\rightarrow t$, si ottiene:
    \begin{equation*}
        \hat U(s,t)\hat U(t,s)=\hat U(s,s)\hat U(s,s)=\hat I,
    \end{equation*}
    che dimostra l'ultima proprietà.
\end{proof}
\begin{remark}
    Si osservi che la prima proprietà implica che il modulo quadro di $\ket{\psi(t)}$ resta costante nel tempo:
    \begin{equation*}
        \braket{\psi(t)|\psi(t)}=\braket{\psi(s)|\hat U^*(t,s)\hat U(t,s)|\psi(s)}=\braket{\psi(s)|\psi(s)}.
    \end{equation*}
\end{remark}
In generale queste due equazioni differenziali non sono di facile risoluzione. Nel caso particolare in cui l'hamiltoniano non dipenda dal tempo queste hanno una semplice soluzione (diventano infatti due equazioni a variabili separabili). Si ha che la soluzione
\begin{equation*}
    \boxed{\hat U(t,s)=e^{-\frac{i}{\hslash}(t-s)\hat H}}\ ,
\end{equation*}
verifica le due equazioni ricavate sopra.
\section{Evoluzione temporale e le rappresentazioni}
Si vogliono ora costruire degli operatori che siano analoghi alle trasformazioni di sistema di riferimento. In generale si desidera che tale operatore ($\hat T$), agendo opportunamente sullo stato del sistema e sugli operatori degli osservabili, mantengano invariati i prodotti scalari (che consentono di ottenere gli osservabili stessi garantendone quindi l'invarianza). Affinché siano preservati i prodotti scalari $\hat T$ deve esse unitario $(\hat T\hat T^*=\hat I$), in questo modo infatti:
\begin{equation*}
    \braket{\psi|\hat T^{*}\hat T|\psi}=\braket{\psi|\psi}.
\end{equation*}
In questo modo si può definire lo stato trasformato $\boxed{\ket{\psi}_T=\hat T^{*}\ket{\psi}}$.\\ Analogamente si può definire il trasformato di un operatore qualsivoglia $\boxed{\hat A_T=\hat T^*\hat A\hat T}$ e in questo modo si ha:
\begin{equation*}
    _T\braket{\psi|\hat A_T|\psi}_T=\braket{\psi|\hat T\hat T^*\hat A\hat T\hat T^*|\psi}=\braket{\psi|\hat A|\psi}.
\end{equation*}
Si può ora richiedere un'ulteriore condizione su questo operatore. Infatti, rendendo dipendente dal tempo $\hat T=\hat T(t,s)$, si può richiedere che soddisfi un set particolare di equazioni differenziali:
\begin{flalign*}
    &i\hslash\frac{\partial\hat T(t,s)}{\partial t}=\hat K(t)\hat T(t,s)\\
    &i\hslash\frac{\partial\hat T(t,s)}{\partial s}=-\hat T(t,s)\hat K(s).
\end{flalign*}
Così facendo (richiedendo anche che $\hat T(t,t)=\hat I$) l'operatore $\hat T$ diviene un fittizio operatore di evoluzione temporale di un hamiltoniano fittizio $\hat K(t)$. \\Il trasformato dello stato diviene $\boxed{\ket{\psi(t)}_K=\hat T^{*}(t,0)\ket{\psi(t)}}$.\\Lo stato trasformato si evolve secondo una nuova equazione di Schrödinger nella quale appare un nuovo hamiltoniano:
\begin{flalign*}
    i\hslash\frac{d}{dt}\ket{\psi(t)}_K&=i\hslash\frac{d}{dt}[\hat T^{*}(t,0)\ket{\psi(t)}]=i\hslash\frac{d\hat T^{*}(t,0)}{dt}\ket{\psi(t)}+i\hslash\hat T^{*}(t,0)\frac{d}{dt}\ket{\psi(t)}\\&=-(\hat K(t)\hat T(t,0))^*\ket{\psi(t)}+\hat T^{*}(t,0)\hat H(t)\ket{\psi(t)}\\&=\hat T^{*}(t,0)[\hat H(t)-\hat K(t)]\hat T\ket{\psi(t)}_K\\
    \Rightarrow&\boxed{\hat H_K(t)=\hat T^{*}(t,0)[\hat H(t)-\hat K(t)]\hat T}\ .
\end{flalign*}
All'hamiltoniano trasformato $\hat H_K$ è associato un operatore di evoluzione temporale $\hat U_K(t,s)$. Si verifica facilmente che $\boxed{\hat U_K=\hat T^*(t,s)\hat U\hat T(t,s)}$.\\
La scelta di $\hat K$ determina la trasformazione che viene chiamata \emph{rappresentazione}.
\subsection{Rappresentazione di Schrödinger}
La rappresentazione più facile da costruire è quella di Schrödinger. 
\begin{equation*}
    \hat K(t)=0\qquad\Rightarrow\quad\hat T(t,s) = \hat I\quad\Rightarrow\qquad
    \begin{cases}
        \ket{\psi(t)}_S=\ket{\psi(t)}\\
        \hat A_S(t)=\hat A(t)\\
        \hat H_S(t)=\hat H(t)\\
        \hat U_S(t,s)=\hat U(t,s)
    \end{cases}.
\end{equation*}
Questa coincide con il formalismo standard che si ottiene senza applicare nessuna trasformazione.
\subsection{Rappresentazione di Heisenberg}
La seconda rappresentazione utilizzata è chiamata rappresentazione di Heisenberg. Questa è determinata dalla scelta $\hat K(t)=\hat H(t)$.
\begin{equation*}
    \hat K(t)=\hat H(t)\qquad\Rightarrow\quad\hat T(t,s) = \hat U(t,s)\quad\Rightarrow\qquad
    \begin{cases}
        \ket{\psi(t)}_H=U^*(t,0)\ket{\psi(t)}=\ket{\psi(0)}\\
        \hat A_H(t)=U^*(t,0)\hat A(t)U(t,0)\\
        \hat H_H(t)=0\\
        \hat U_H(t,s)=\hat I
    \end{cases}.
\end{equation*}
In questo caso si osservi che la rappresentazione di Heisenberg rende nulla l'hamiltoniano. L'evoluzione del sistema è interamente determinata dall'evoluzione degli operatori che rappresentano gli osservabili (tanto è vero che pure gli stati diventano indipendenti dal tempo). Questo tipo di rappresentazione è analoga al cambio di sistema di riferimento che, in meccanica classica, rende l'osservatore solidale ad un corpo in moto.
\subsection{Rappresentazione di Dirac}
L'ultima rappresentazione è la rappresentazione di Dirac o rappresentazione di interazione. Questa è particolarmente utile quando l'hamiltoniano del sistema è dato da un termine indipendente dal tempo sommato ad un secondo termine dipendente dal tempo (solitamente una perturbazione):
\begin{equation*}
    \hat H(t)=\hat H_0+\hat W(t).
\end{equation*}
La rappresentazione di Dirac rimuove il termine indipendente dal tempo richiedendo che l'hamiltoniano fittizio sia proprio tale termine.
\begin{equation*}
    \hat K(t)=\hat H_0\quad\Rightarrow\quad\hat T(t,s) =\hat U_0(t,s)= e^{-\frac{i}{\hslash}(t-s)\hat H_0}\quad\Rightarrow\quad
    \begin{cases}
        \ket{\psi(t)}_D=U_0^*(t,0)\ket{\psi(t)}\\
        \hat A_D(t)=U_0^*(t,0)\hat A(t)U_0(t,0)\\
        \hat H_D(t)=U_0^*(t,0)\hat W(t)U_0(t,0)=\hat W_D(t)\\
        \hat U_D(t,s)=U_0^*(t,0)\hat U(t,s)U_0(t,0)
    \end{cases}.
\end{equation*}