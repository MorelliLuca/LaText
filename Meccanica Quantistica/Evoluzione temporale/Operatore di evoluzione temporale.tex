\section{Equazione di Schrödinger}
Come si è visto, uno stato quantomeccanico consente di ottenere gli osservabili del sistema a cui appartiene attraverso l'uso di operatori autoaggiunti. Inoltre si è sempre fatto riferimento ad uno stato definito in un istante e di cui si vogliono misurare alcuni osservabili. Si vuole ora descrivere l'evoluzione temporale di tale stato.\\

Sia $\ket{\psi_0}$ lo stato del sistema ad un istante iniziale $t=0$, si indicherà con $\ket{\psi(t)}$ il suo stato ad un istante successivo $t$. L'evidenza sperimentale mostra che gli stati $\ket{\psi(t)}$ sono soluzione della nota \emph{equazione di Schrödinger}:
\begin{equation}
    i\hslash\frac{d\ket{\psi(t)}}{dt}=\hat H(t)\ket{\psi(t)},\label{eqSchr}
\end{equation}
dove $\hat H(t)$ è l'operatore hamiltonano, ovvero l'operatore associato all'energia del sistema in un istante $t$. Imponendo $\ket{\psi(0)}=\ket{\psi_0}$ si ottiene il problema di Cauchy che ha come soluzione l'evoluzione temporale dello stato.
\section{L'operatore di evoluzione temporale}
Si osservi che l'equazione di Schrödinger \eqref{eqSchr} è un'equazione lineare, questo ha due dirette conseguenze:
\begin{itemize}
    \item se così non fosse non varrebbe il Principio di sovrapposizione,
    \item si può esprimere ogni stato come combinazione lineare di altri stati precedenti (quindi esiste un operatore lineare che fa questa operazione).
\end{itemize}
Si desidera costruire un operatore che descriva l'evoluzione del sistema associando ad uno stato iniziale il suo evoluto.
\begin{definition}
    Si definisce $\hat U(t,0)$ \emph{operatore di evoluzione temporale} se dato uno stato $\ket{\psi(t)}$:
    \begin{equation*}
        \ket{\psi(t)}=\hat U(t,s)\ket{\psi(s)}.
    \end{equation*}
\end{definition}