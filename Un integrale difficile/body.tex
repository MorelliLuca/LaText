\begin{center}
	\huge \textbf{Un integrale complesso}
	
	\rule{7cm}{0.4pt} 
	
	\LARGE 
	
	\vspace{10pt}
	
	\LARGE \textbf{Luca Morelli}
	
	\vspace{5pt}
	
	\LARGE 20 Dicembre 2022
	
\end{center}

\vspace{20pt}




	\section{Introduzione}
	Si vuole calcolare il valore del seguente integrale
	\begin{equation}
		\int_{-\infty}^{\infty} \frac{x^2}{(1+e^x)(1+e^{-x})}dx
		\label{int}
	\end{equation}
    \section{Il calcolo}
	Osserviamo che l'integrando in (\ref{int}) è pari per cui è possibile 
	modificare il dominio di integrazione nel intervallo $[0,\infty)$. Inoltre 
	possiamo effettuare la seguente sotituzione: $x=\log z$.
	\begin{equation*}
		\int_{-\infty}^{\infty} \frac{x^2}{(1+e^x)(1+e^{-x})}dx=2\int_{0}^{\infty} \frac{x^2}{(1+e^x)(1+e^{-x})}dx=
		2\int_{1}^{\infty} \frac{\log^2(z)}{(1+z)^2}dz
	\end{equation*}
	
	Considerando ora un cammino di integrazione nel piano complesso composto da 
	una circonferenza $\Gamma$ congiunta con due segmenti posizionati uno appena 
	al disotto e uno appena sopra l'asse reale che corrono nell'intervallo $[1,\infty)$, e facendo tendere il raggio di $\Gamma$ ad infinito  
    è possibile sfruttare il Teorema di Residui:
	\begin{flalign}\label{log2}
		\int_{1}^{\infty} \frac{\log^2(z)}{(1+z)^2}dz&=2\pi i\sum_{poli} Res\frac{\log^2(z)}{(1+z)^2}+\int_{1}^{\infty} \frac{(\log(z)+2\pi i)^2}{(1+z)^2}dz\\ \nonumber
        &=2\pi i\sum_{poli} Res\frac{\log^2(z)}{(1+z)^2}+\int_{1}^{\infty} \frac{\log^2(z)}{(1+z)^2}dz+4\pi i\int_{1}^{\infty}\frac{\log(z)}{(1+z)^2}dz-4\pi^2\int_{1}^{\infty}\frac{1}{(1+z)^2}dz
	\end{flalign}
    Questa relazione però non consente di calcolare l'integrale desiderato infatti se 
	proseguiamo a sviluppare il quadrato presente nel secondo integrale osserviamo che 
	l'integrale desiderato si elide. Calcoliamo quindi il seguente integrale:
	\begin{flalign}\label{log3}
		\int_{1}^{\infty} \frac{\log^3(z)}{(1+z)^2}dz&=2\pi i\sum_{poli} Res\frac{\log^3(z)}{(1+z)^2}+\int_{1}^{\infty} \frac{(\log(z)+2\pi i)^3}{(1+z)^2}dz\\ \nonumber
		&=2\pi i\sum_{poli} Res\frac{\log^3(z)}{(1+z)^2}+\int_{1}^{\infty} \frac{\log(z)^3}{(1+z)^2}dz + 6\pi i \int_{1}^{\infty}\frac{\log(z)^2}{(1+z)^2}dz +\\ \nonumber
		&\qquad\qquad\qquad\qquad- 12\pi^2\int_{1}^{\infty}\frac{\log(z)}{(1+z)^2}dz -8\pi^3i \int_{1}^{\infty}\frac{1}{(1+z)^2}dz 
	\end{flalign}\\

	Per prima cosa valutiamo l'ultimo integrale che abbiamo trovato nelle (\ref{log2}) (\ref{log3}), procediamo con un semplice cambio di variabili:
	\begin{equation}
		\int_{1}^{\infty}\frac{1}{(1+x)^2}dx=\int_{2}^{\infty}\frac{1}{u^2}du =-\frac{2}{u}\bigg|_2^\infty= 1
		\label{easyint}
	\end{equation} \\

	Dobbiamo quindi calcolare i residui integrali delle relazioni (\ref{log2}) (\ref{log3}), entrambe le funzioni di cui si devono calcolare i residui 
	presentano un polo in $z=-1$. Osserviamo però che dalla definizione di residuo 
	e dall'espressione della derivata n-esima di una funzione olomorfa si ha:
	\begin{equation}
		\label{trickfigo}
		2\pi i\ Res\frac{f(z)}{(1+z)^2}\bigg|_{z_0}=\oint_\gamma\frac{f(z)}{(1+z)^2}dz=2\pi i\ \frac{df}{dz}(z)\bigg|_{z=-1}
	\end{equation}
	dove $\gamma$ è una curva chiusa regolare che racchiude sia il punto $z_0$ che il punto $z=-1$.\\
    Se $f(z)=\frac{\log^2(z)}{(1+z)^2}$ allora, considerando la presenza di due poli, si ha:
		\begin{equation}\label{res1}
			\sum_{poli} Res\frac{\log^2(z)}{(1+z)^2}=\frac{d}{dz} \log^2(z) \bigg|_{z=-1}=
            2\frac{\log(z)}{z}\bigg|_{z=-1}=-2\pi i
		\end{equation}
	Con analoghe considerazioni si ha per $f(z)=\frac{\log^3(z)}{(1+z)^2}$:
	\begin{equation}\label{res2}
		\sum_{poli} Res\frac{\log^3(z)}{(1+z)^2}= \frac{d}{dz} \log^3(z) \bigg|_{z=-1}=
		3\frac{\log^2(z)}{z}\bigg|_{z=-1}=3\pi^2
	\end{equation}
	Sostituendo nella (\ref{log2}) i risultati appena ottenuti (\ref{easyint}) (\ref{res1}) ed elidendo i termini ripetuti abbiamo:
	\begin{equation}
        \int_{1}^{\infty}\frac{\log(z)}{(1+z)^2}dz=\pi i-\pi i=0
		\label{log}
	\end{equation}
    Possiamo quindi sostituire nella (\ref{log3}) i risultati della (\ref{easyint}), (\ref{log3}) e (\ref{log}) ottenendo:
	\begin{equation}
		\int_{1}^{\infty}\frac{\log(z)^2}{(1+z)^2}dz=\frac{4}{3}\pi^2-\pi^2=\frac{\pi^2}{3}
	\end{equation} 




    