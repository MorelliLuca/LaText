\section{The second quantization}
In the discussion of relativistic quantum mechanics (Part \ref{part:RQM}) we have derived the Klein-Gordon equation
\begin{equation*}
    (\Box +m^2)\varphi\big(x^\mu\big)=0
\end{equation*}
from the first quantization of special relativity. In this section we will attempt the second quantization procedure of this equation, in the same manner we described introducing quantum field theory.\\

As we already mentioned in the previous section, the lagrangian density of the Klein-Gordon equation is given by:
\begin{equation}\label{KGLagrDensity}
    \mathcal{L}=\frac{1}{2}\partial_\mu\partial^\mu\varphi-\frac{m^2}{2}\varphi^2,
\end{equation}
where $m$ is the mass of a single particle described by the field $\varphi$.\\
We now need to quantize the fields by promoting them to operators: these operators will be some functions of space but not of time, since the time evolution will be generated by a Schrödinger equation.
We remind that in classical quantum mechanics we promote position and momentum to operators imposing the commutation relations given by the Poisson brackets
\begin{equation*}
    q_i,p_j=\frac{\partial L}{\partial \dot q_j}\quad\longrightarrow\quad \hat{q}_i, \hat{p}_j\quad \text{with}\quad [q_i,p_j]=i\delta_{ij}.
\end{equation*}
In order to promote the field $\varphi$, and its conjugate field $\pi=\frac{\partial\mathcal{L} }{\partial\partial_0\varphi}$, we will impose similar commutation relations:
\begin{equation*}
    \varphi_i,\pi^j\quad\longrightarrow\quad \hat{\varphi}_i, \hat{\pi}^j\quad \text{with}\quad [\hat{\varphi}_i(\vec x), \hat{\pi}^j(\vec y)]=i\delta^3(\vec x-\vec y)\delta_{i}^j.
\end{equation*}
As we have done with the elastic string, we have to decouple all the harmonic oscillators that arise from the second derivative in time in the Klein Gordon equation. To do so we use the Fourier Transform in the differential equation itself:
\begin{equation*}
    \varphi(\vec x,t)=\int \frac{d^3p}{(2\pi)^3}e^{i\vec p\cdot \vec x} \tilde\varphi(\vec p,t) \quad \Rightarrow\quad \int \frac{d^3p}{(2\pi)^3}e^{i\vec p\cdot \vec x}\bigg(\frac{\partial^2}{\partial t^2}+\vec p^2+m^2\bigg)\tilde\varphi(\vec p,t)=0.
\end{equation*}
In this way we have recovered an infinite series of harmonic oscillators, characterized by $\omega_{\vec{p}}=\sqrt{\vec p^2+m^2}$, which are the Fourier transform of the solutions $\tilde\varphi$. In this way every harmonic oscillator will be quantized by its own creation and annihilation operators:
\begin{equation*}
    \begin{cases}
        \hat q=\frac{1}{\sqrt{2\omega}}\big(\hat{a}+\hat{a}^\dagger\big)\\
        \hat p=-i\sqrt{\frac{\omega}{2}}\big(\hat{a}-\hat{a}^\dagger\big)
    \end{cases}\quad\Rightarrow\quad
    \begin{cases}
        \hat{\tilde{\varphi}}^{\vec{p}}=\frac{1}{\sqrt{2\omega_{\vec{p}}}}\big(\hat{a}_{\vec{p}}+\hat{a}_{\vec{p}}^\dagger\big)\\
        \hat{\tilde{\pi}}_{\vec{p}}=-i\sqrt{\frac{\omega_{\vec{p}}}{2}}\big(\hat{a}_{\vec{p}}-\hat{a}_{\vec{p}}^\dagger\big)
    \end{cases}.
\end{equation*}
These creation and annihilation operators must satisfy their own commutation relations (which are just generalizations of the classical ones):
\begin{equation*}
    [\hat{a}_{\vec{p}},\hat{a}_{\vec{q}}]=[\hat{a}^{\dagger}_{\vec{p}},\hat{a}_{\vec{q}}^{\dagger}]=0\quad \text{and}\quad [\hat{a}_{\vec{p}},\hat{a}_{\vec{q}}^{\dagger}]=(2\pi)^3\delta^3(\vec p-\vec q).
\end{equation*}
By the Fourier transform we then get the two field operators as:
\begin{equation*}
        \hat{\varphi}=\int\frac{d^3p}{(2\pi)^3}\frac{e^{i\vec p\cdot \vec x}}{\sqrt{2\omega_{\vec{p}}}}\big(\hat{a}_{\vec{p}}+\hat{a}_{\vec{p}}^\dagger\big),\qquad
        \hat{\pi}=-i\int\frac{d^3p}{(2\pi)^3}\sqrt{\frac{\omega_{\vec{p}}}{2}}e^{i\vec p\cdot \vec x}\big(\hat{a}_{\vec{p}}-\hat{a}_{\vec{p}}^\dagger\big),
\end{equation*}
that, since the integration is over the whole momenta space and the solutions are the same for $\vec p$ or $-\vec p$ (they depend on its modulus), can be rewritten as
\begin{equation}\label{KGFieldOperators}
    \hat{\varphi}=\int\frac{d^3p}{(2\pi)^3}\frac{1}{\sqrt{2\omega_{\vec{p}}}}\bigg(\hat{a}_{\vec{p}}e^{i\vec p\cdot \vec x}+\hat{a}_{\vec{p}}^\dagger e^{-i\vec p\cdot \vec x}\bigg),\qquad
    \hat{\pi}=-i\int\frac{d^3p}{(2\pi)^3}\sqrt{\frac{\omega_{\vec{p}}}{2}}\bigg(\hat{a}_{\vec{p}}e^{i\vec p\cdot \vec x}-\hat{a}_{\vec{p}}^\dagger e^{-i\vec p\cdot \vec x}\bigg).
\end{equation}
We can verify that these satisfy the commutation conditions with a simple calculation:
\begin{align*}
    [\hat{\varphi}_i(\vec x), \hat{\pi}^j(\vec y)]&=\frac{-i}{2}\int\frac{d^3p\ d^3q}{(2\pi)^6}\sqrt{\frac{\omega_{\vec{q}}}{\omega_{\vec{p}}}} \bigg[\hat{a}_{\vec{p}}e^{i\vec p\cdot \vec x}+\hat{a}_{\vec{p}}^\dagger e^{-i\vec p\cdot \vec x},\hat{a}_{\vec{q}}e^{i\vec q\cdot \vec x}-\hat{a}_{\vec{q}}^\dagger e^{-i\vec q\cdot \vec x}\bigg]\\&=\frac{i}{2}\int\frac{d^3p\ d^3q}{(2\pi)^6}\sqrt{\frac{\omega_{\vec{q}}}{\omega_{\vec{p}}}} (2\pi)^3\delta^3(\vec p-\vec q)\bigg\{e^{i(\vec p\cdot \vec x-\vec q\cdot\vec y)}+e^{-i(\vec p\cdot \vec x-\vec q\cdot\vec y)}\bigg\}\\&=\frac{i}{2}\int\frac{d^3p}{(2\pi)^3}\bigg\{e^{i(\vec p\cdot \vec x-\vec p\cdot\vec y)}+e^{-i(\vec p\cdot \vec x-\vec p\cdot\vec y)}\bigg\}=i\delta(\vec x-\vec y).
\end{align*}
We now want to find the hamiltonian of the system, in order to quantize it:
\begin{align*}
    \mathcal{H} &=\int d^3x (\pi\partial_t\varphi-\mathcal{L})=\int d^3x \bigg(\pi^2-\frac{1}{2}\partial_\mu\partial^\mu\varphi-\frac{m^2}{2}\varphi^2\bigg)\\&=\int d^3x \bigg(\pi^2-\frac{1}{2}\pi^2+\frac{1}{2}(\vec\nabla\varphi)^2-\frac{m^2}{2}\varphi^2\bigg)\\&=\int d^3x \bigg(\frac{1}{2}\pi^2+\frac{1}{2}(\vec\nabla\varphi)^2-\frac{m^2}{2}\varphi^2\bigg).
\end{align*}
The quantization procedure consists in the substitution of the fields with their operational counterparts, that we have previously derived:
\begin{align*}
    \hat{\mathcal{H}} &=\int d^3x \bigg(\frac{1}{2}\hat{\pi}^2+\frac{1}{2}(\vec\nabla\hat{\varphi})^2-\frac{m^2}{2}\hat{\varphi}^2\bigg)\\&=\frac{1}{2}\int\frac{d^3x\ d^3p\ d^3q}{(2\pi)^6}\bigg[-\frac{\sqrt{\omega_{\vec{p}}\omega_{\vec{q}}}}{2}\bigg(\hat{a}_{\vec{p}}e^{i\vec p\cdot \vec x}-\hat{a}_{\vec{p}}^\dagger e^{-i\vec p\cdot \vec x}\bigg)\bigg(\hat{a}_{\vec{q}}e^{i\vec q\cdot \vec x}-\hat{a}_{\vec{q}}^\dagger e^{-i\vec q\cdot \vec x}\bigg)+\\&-\frac{\vec p\cdot\vec q}{2\sqrt{\omega_{\vec p}\omega_{\vec q}}}\bigg(\hat{a}_{\vec{p}}e^{i\vec p\cdot \vec x}-\hat{a}_{\vec{p}}^\dagger e^{-i\vec p\cdot \vec x}\bigg)\bigg(\hat{a}_{\vec{q}}e^{i\vec q\cdot \vec x}-\hat{a}_{\vec{q}}^\dagger e^{-i\vec q\cdot \vec x}\bigg)+\frac{m^2}{2\sqrt{\omega_{\vec p}\omega_{\vec q}}}\times\\&\times\bigg(\hat{a}_{\vec{p}}e^{i\vec p\cdot \vec x}+\hat{a}_{\vec{p}}^\dagger e^{-i\vec p\cdot \vec x}\bigg)\bigg(\hat{a}_{\vec{q}}e^{i\vec q\cdot \vec x}+\hat{a}_{\vec{q}}^\dagger e^{-i\vec q\cdot \vec x}\bigg)\bigg] .
\end{align*}
This integral can be solved with ease if we integrate first with respect to $x$, using the Fourier representation of the Dirac's delta function:
\begin{align*}
    &\frac{1}{2}\int\frac{d^3p\ d^3q}{(2\pi)^3}\bigg[\delta^3(\vec p+\vec q)\big(\hat{a}_{\vec{p}}\hat{a}_{\vec{q}}+\hat{a}_{\vec{p}}^\dagger\hat{a}_{\vec{q}}^\dagger\big)\bigg(-\frac{\sqrt{\omega_{\vec{p}}\omega_{\vec{q}}}}{2}-\frac{\vec p\cdot\vec q}{2\sqrt{\omega_{\vec p}\omega_{\vec q}}}+\frac{m^2}{2\sqrt{\omega_{\vec p}\omega_{\vec q}}}\bigg)+\\ &\qquad-\delta^3(\vec p-\vec q)\big(\hat{a}_{\vec{p}}\hat{a}^\dagger_{\vec{q}}+\hat{a}^\dagger_{\vec{p}}\hat{a}_{\vec{q}}\big)\bigg(-\frac{\sqrt{\omega_{\vec{p}}\omega_{\vec{q}}}}{2}-\frac{\vec p\cdot\vec q}{2\sqrt{\omega_{\vec p}\omega_{\vec q}}}-\frac{m^2}{2\sqrt{\omega_{\vec p}\omega_{\vec q}}}\bigg)\bigg]\\
    &=\frac{1}{2}\int\frac{d^3p}{(2\pi)^3}\bigg[\big(\hat{a}_{\vec{p}}\hat{a}_{\vec{p}}+\hat{a}_{\vec{p}}^\dagger\hat{a}_{\vec{p}}^\dagger\big)\bigg(-\frac{\omega_{\vec{p}}}{2}+\frac{\vec p^2}{2\omega_{\vec p}}+\frac{m^2}{2\omega_{\vec p}}\bigg)+\\&\qquad -\big(\hat{a}_{\vec{p}}\hat{a}^\dagger_{\vec{p}}+\hat{a}^\dagger_{\vec{p}}\hat{a}_{\vec{p}}\big)\bigg(-\frac{\omega_{\vec{p}}}{2}-\frac{\vec p^2}{2\omega_{\vec p}}-\frac{m^2}{2\omega_{\vec p}}\bigg)\bigg].
\end{align*} 
We should now notice that the first term of the integrand vanishes since
\begin{equation*}
    -\frac{\omega_{\vec{p}}}{2}+\frac{\vec p^2}{2\omega_{\vec p}}+\frac{m^2}{2\omega_{\vec p}}=\frac{1}{2\omega_{\vec p}}(-\omega_{\vec{p}}^2+\vec{p}^2+m^2)=\frac{1}{2\omega_{\vec p}}(-\vec{p}^2-m^2+\vec{p}^2+m^2)=0,
\end{equation*}
and thus the integral reads:
\begin{equation*}
    \frac{1}{2}\int\frac{d^3p}{(2\pi)^3}\big(\hat{a}_{\vec{p}}\hat{a}^\dagger_{\vec{p}}+\hat{a}^\dagger_{\vec{p}}\hat{a}_{\vec{p}}\big)\bigg(\frac{\omega_{\vec{p}}}{2}+\frac{\vec p^2}{2\omega_{\vec p}}+\frac{m^2}{2\omega_{\vec p}}\bigg)=\frac{1}{2}\int\frac{d^3p}{(2\pi)^3}\omega_{\vec{p}}\big(\hat{a}_{\vec{p}}\hat{a}^\dagger_{\vec{p}}+\hat{a}^\dagger_{\vec{p}}\hat{a}_{\vec{p}}\big).
\end{equation*}
Using the commutator  for the creation annihilation operator we can manipulate the last expression
\begin{equation*}
    \hat{a}_{\vec{p}}\hat{a}^\dagger_{\vec{p}}+\hat{a}^\dagger_{\vec{p}}\hat{a}_{\vec{p}}=[\hat{a}_{\vec{p}},\hat{a}^\dagger_{\vec{p}}]+2\hat{a}^\dagger_{\vec{p}}\hat{a}_{\vec{p}}=2\hat{a}^\dagger_{\vec{p}}\hat{a}_{\vec{p}}+(2\pi)^3\delta^3(0),
\end{equation*}
therefore the hamiltonian operator reads:
\begin{equation}\label{KGHamiltonianOp}
    \hat{\mathcal{H} }=\frac{1}{2}\int\frac{d^3p}{(2\pi)^3}\omega_{\vec{p}}\hat{a}^\dagger_{\vec{p}}\hat{a}_{\vec{p}}+\int d^3p\omega_{\vec{p}}\delta^3(0)
\end{equation}
The first term that appears in the equation \eqref{KGHamiltonianOp} is responsible for the energy contribution of each particle in the system, while the second one represent the energy of the vacuum. It is easy to see that in fact the eigenvalue of $\ket{0}$ is non-zero (as in classical quantum mechanics):
\begin{equation*}
    \hat{\mathcal{H} }\ket{0}=\frac{1}{2}\int\frac{d^3p}{(2\pi)^3}\omega_{\vec{p}}\hat{a}^\dagger_{\vec{p}}\hat{a}_{\vec{p}}\ket{0}+\int d^3p\omega_{\vec{p}}\delta^3(0)\ket{0}=\int d^3p\omega_{\vec{p}}\delta^3(0)\ket{0}.
\end{equation*}   
This term is actually divergent for two reasons:
\begin{itemize}
    \item $\delta^3(0)$ (\textbf{IR Divergence}) is some kind of infinity and it is due to the fact that vacuum has energy and the space we are considering is infinite, this can be shown considering Fourier representation of the delta
    \begin{equation*}
        d^3(0)=\lim_{L\rightarrow\infty}\int_{L/2}^{L/2}d^3x\ e^{i\vec p\cdot\vec x}\bigg|_{\vec p=0}=\lim_{L\rightarrow\infty}L^3
    \end{equation*}
    and in this way we can treat this divergence restricting the spacial domain of our system to a box;
    \item $\int d^3p\omega_{\vec{p}}\rightarrow\infty$ (\textbf{UV Divergence}), this divergence is due to incostincencies of the theory at high energies probably due to gravitational interactions, this last type of divergence is treated by introducing a \emph{cut-off} where the theory stops to work. 
\end{itemize}
Furthermore, we could define the \textbf{normal ordering} hamiltonian
\begin{equation*}
    :\hat{\mathcal{H} }:=\hat{\mathcal{H} }-\bra{0}\hat{\mathcal{H}} \ket{0},
\end{equation*}
this one removes the divergent part of the hamiltonian leaving just energies meant as differences between the vacuum energy and the one of the system. This hamiltonian can be obtained by manipulation of the classical hamiltonian, before quantization.\\

Lastly, let's consider the state
\begin{equation*}
    \ket{\vec{p}}=\hat{a}^\dagger_{\vec p}\ket{0},
\end{equation*} 
letting the hamiltonian act on this (in normal ordering) we get:
\begin{align*}
    \hat{\mathcal{H} }\ket{\vec{p}}&=\frac{1}{2}\int\frac{d^3q}{(2\pi)^3}\omega_{\vec{q}}\hat{a}^\dagger_{\vec{q}}\hat{a}_{\vec{q}}\hat{a}^\dagger_{\vec{p}}\ket{0}=\frac{1}{2}\int\frac{d^3q}{(2\pi)^3}\omega_{\vec{q}}\hat{a}^\dagger_{\vec{q}}([\hat{a}_{\vec{q}},\hat{a}^\dagger_{\vec{p}}]+\hat{a}^\dagger_{\vec{p}}\hat{a}_{\vec{q}})\ket{0}\\&=\frac{1}{2}\int\frac{d^3q}{(2\pi)^3}\omega_{\vec{q}}\hat{a}^\dagger_{\vec{q}}[\hat{a}_{\vec{q}},\hat{a}^\dagger_{\vec{p}}]\ket{0}=\frac{1}{2}\int\frac{d^3q}{(2\pi)^3}\omega_{\vec{q}}\hat{a}^\dagger_{\vec{q}}\ket{0}(2\pi)^2\delta^3(\vec p-\vec q)\\&=\omega_{\vec{p}}\hat{a}^\dagger_{\vec{q}}\ket{0},
\end{align*}
which shows that the acting of the creation operator on the vacuum correspond to the creation of a particle.
The same procedure can be used to show that the momentum operator $\hat{\vec{p}}$ acting on $\ket{\vec p}$ returns an eigenvalue of $\vec p$:
\begin{equation*}
    \hat{\vec{p}}=-\int d^3x\hat{\pi}\vec{\nabla}\hat{\varphi}=\int\frac{d^3p}{(2\pi)^3}\vec{p}\hat{a}_{\vec{p}}^\dagger\hat{a}_{\vec{p}}.
\end{equation*}
The same is true for the number operator 
\begin{equation*}
    \hat{N}=\int\frac{d^3p}{(2\pi)^3}\hat{a}_{\vec{p}}^\dagger\hat{a}_{\vec{p}},
\end{equation*}
which returns the total number of particles in the Fock space. Whiteout interactions $\hat N$ commutes with the hamiltonian and therefore the number of particles is conserved. We should now notice that, being all the creation operators commuting, if they create particles with different momentum, all the states that can be described by this theory are symmetric with respect to exchange of two particles:
\begin{equation*}
    \ket{\vec q, \vec p}=\hat{a}^\dagger_{\vec{q}}\hat{a}^\dagger_{\vec{p}}\ket{0}=\hat{a}^\dagger_{\vec{p}}\hat{a}^\dagger_{\vec{q}}\ket{0}=\ket{\vec p,\vec q}.
\end{equation*}
We should now notice that this formalism is not Lorentz covariant, in fact, if we study the completeness relation, which should hold in every reference frame, it is easy to see that it is made up of to non-Lorentz invariant parts:
\begin{equation*}
    \hat{\mathds{1}}=\int\frac{d^3p}{(2\pi)^3}\ket{\vec p}\bra{\vec p}.
\end{equation*}
The natural choice to get an invariant internal expression is to integrate over $d^4P$, which transform with the absolute value of the determinant of a Lorentz transformation, which is $1$, thus $d^4P$ is actually Lorentz invariant.\\We need to reduce this integral to the previous one, this can be done using a single Dirac's delta that fixes the energy $P^0=\sqrt{\vec p^2+m^2}$, paying attention not to include negative energies (using a $\theta$ Heaviside function):
\begin{equation*}
    \hat{\mathds{1}}=\int\frac{d^4P}{(2\pi)^3}\delta((P^0)^2-\vec p^2-m^2)\theta(P^0)\ket{\vec p}\bra{\vec p}.
\end{equation*}
The delta function can be manipulated resulting in:
\begin{equation*}
    \hat{\mathds{1}}=\int\frac{d^3p}{(2\pi)^3E_{\vec p}}\ket{\vec p}\bra{\vec p},
\end{equation*}
that can have the same form of the original completeness relation by scaling every state by its energy $\ket{p}=\sqrt{E_{\vec p}}\ket{\vec p}$.
\section{Charges and symmetries of the field}
We will now study a system made up of $2$ Klein-Gordon real fields, then we will see that symmetries in the lagrangian of these two fields lead naturally to the finding of some discrete conserved charge, that can be interpreted as an electric charge.\\

The lagrangian of a system of two Klein-Gordon field is composed by the sum of the lagrangian of both fields alone:
\begin{equation*}
    \mathcal{L}=\sum_{i=1}^{2}\bigg(\frac{1}{2}\partial_\mu\varphi_i\partial^\mu\varphi_i-\frac{m_i}{2}\varphi_i^2\bigg).
\end{equation*}
This lagrangian will give raise to two different (due to the different masses) and independent fields, as solutions of two different and independent Klein-Gordon equations. Each system can be quantized, as we already have done, defining its own set of creation/annihilation operators that will result in two distinct hamiltonian, 3-momentum and number operators (one for each field):
\begin{equation*}
    \hat{\mathcal{H} }_i=\int\frac{d^3p}{(2\pi)^3}\omega_{i,\vec p}\hat{a}^{\dagger}_{i,\vec p}\hat{a}_{i,\vec p},\quad\hat{\vec p}_i=\int\frac{d^3p}{(2\pi)^3}\vec p_{i}\hat{a}^{\dagger}_{i,\vec p}\hat{a}_{i,\vec p},\quad\hat{N}_i=\int\frac{d^3p}{(2\pi)^3}\omega_{i,\vec p}\hat{a}^{\dagger}_{i,\vec p}\hat{a}_{i,\vec p}.
\end{equation*}
From these we can build their total version, accounting for particles of both the fields:
\begin{equation*}
    \hat{\mathcal{H} }=\hat{\mathcal{H} }_1+\hat{\mathcal{H} }_2,\quad\hat{\vec p }=\hat{\vec p }_1+\hat{\vec p}_2,\quad\hat{N}=\hat{N}_1+\hat{N}_2.
\end{equation*}  
In this construction we don't have degenerates, since two particles with the same momentum will have different energies, and so different hamiltonian eigenvalues, and their state will be in different eigenspace.\\ The same is not true anymore if we consider to fields with the same mass. In this case, even though the system is now degenerate, the lagrangian acquires a new symmetry. In fact, we can think to those two fields as a $2$-D real vector $\vec\varphi$:
\begin{equation}\label{KGLagrangeDensity2F}
    \mathcal{L} =\frac{1}{2}\partial_\mu\vec\varphi^T \partial^\mu\vec\varphi-\frac{m}{2}\vec\varphi^T\vec\varphi,
\end{equation}
it is now clear that all the rotation in the space of $\vec\varphi$ won't change the lagrangian. Therefore, the Lagrangian has a symmetry of the group $SO(2)$ that we can exploit in order to use the Nother's theorem.\\
Let's consider an infinitesimal rotation:
\begin{equation*}
    R=\begin{pmatrix}
        \cos\theta&&\sin\theta\\-\sin\theta&&\cos\theta
    \end{pmatrix}\quad\Rightarrow\quad\delta R=\begin{pmatrix}
        1&&\theta\\-\theta&&1
    \end{pmatrix}=\mathds{1}+=\begin{pmatrix}
        0&&\theta\\-\theta&&0
    \end{pmatrix},
\end{equation*}
acting on the fields this gives us the variations
\begin{equation*}
    \begin{pmatrix}
        \varphi_1'\\\varphi_2'
    \end{pmatrix}=\begin{pmatrix}
        1&&\theta\\-\theta&&1
    \end{pmatrix}\begin{pmatrix}
        \varphi_1\\\varphi_2
    \end{pmatrix}=\begin{pmatrix}
        \varphi_1+\theta\varphi_2\\\varphi_2-\theta\varphi_1
    \end{pmatrix}\quad\Rightarrow\quad\begin{cases}
        \delta\varphi_1=\theta\varphi_2\\
        \delta\varphi_2=-\theta\varphi_1
    \end{cases}.
\end{equation*}
Now, Nother's theorem gives us a conserved current:
\begin{equation*}
    J^\mu=\frac{\partial\mathcal{L} }{\partial\partial_\mu \varphi_i}\delta\varphi_i=\partial^\mu\varphi_1\theta\varphi_2-\partial^\mu\varphi_2\theta\varphi_1.
\end{equation*}
Neglecting $\theta$ (which is just a constant multiplicative factor), we can get a conserved charge from the first component of the current:
\begin{equation*}
    Q=\int d^3x\ J^0=\int d^3x\ (\dot\varphi_1\varphi_2-\dot\varphi_2\varphi_1),\qquad\frac{dQ}{dt}=0.
\end{equation*}
We now have to quantize this observable, this can be done plugging in the quantized fields \eqref{KGFieldOperators}:
\begin{align*}
   &\hat Q=\int d^3x\ (\hat\pi_1\hat\varphi_2-\hat\pi_2\hat\varphi_1)\\
   &\int d^3x\ \hat\pi_i\hat\varphi_j=\frac{-i}{2}\int\frac{d^3x\ d^3p\ d^3q}{(2\pi)^6}\sqrt{\frac{\omega_{\vec{q}}}{\omega_{\vec{p}}}} \bigg(\hat{a}_{i,\vec{q}}e^{i\vec q\cdot \vec x}-\hat{a}_{i,\vec{q}}^\dagger e^{-i\vec q\cdot \vec x}\bigg)\bigg(\hat{a}_{j,\vec{p}}e^{i\vec p\cdot \vec x}+\hat{a}_{j,\vec{p}}^\dagger e^{-i\vec p\cdot \vec x}\bigg)\\&=\frac{-i}{2}\int\frac{d^3p\ d^3q}{(2\pi)^3}\sqrt{\frac{\omega_{\vec{q}}}{\omega_{\vec{p}}}}\bigg[(\hat{a}_{i,\vec{q}}\hat{a}_{j,\vec{p}}-\hat{a}_{i,\vec{q}}^\dagger\hat{a}_{j,\vec{p}}^\dagger)\delta^3(\vec p+\vec q)+(\hat{a}_{i,\vec{q}}\hat{a}_{j,\vec{p}}^\dagger-\hat{a}_{i,\vec{q}}^\dagger\hat{a}_{j,\vec{p}}) \delta^3(\vec p-\vec q) \bigg]\\&=\frac{-i}{2}\int\frac{d^3p}{(2\pi)^3}\bigg[(\hat{a}_{i,-\vec{p}}\hat{a}_{j,\vec{p}}-\hat{a}_{i,-\vec{p}}^\dagger\hat{a}_{j,\vec{p}}^\dagger)+(\hat{a}_{i,\vec{p}}\hat{a}_{j,\vec{p}}^\dagger-\hat{a}_{i,\vec{p}}^\dagger\hat{a}_{j,\vec{p}})  \bigg]\\
   &\hat{Q}=\frac{-i}{2}\int\frac{d^3p}{(2\pi)^3}\bigg[(\hat{a}_{1,-\vec{p}}\hat{a}_{2,\vec{p}}-\hat{a}_{1,-\vec{p}}^\dagger\hat{a}_{2,\vec{p}}^\dagger)+(\hat{a}_{1,\vec{p}}\hat{a}_{2,\vec{p}}^\dagger-\hat{a}_{1,\vec{p}}^\dagger\hat{a}_{2,\vec{p}})+\\&\qquad\qquad\qquad\qquad-(\hat{a}_{2,-\vec{p}}\hat{a}_{1,\vec{p}}-\hat{a}_{2,-\vec{p}}^\dagger\hat{a}_{1,\vec{p}}^\dagger)-(\hat{a}_{2,\vec{p}}\hat{a}_{1,\vec{p}}^\dagger-\hat{a}_{2,\vec{p}}^\dagger\hat{a}_{1,\vec{p}})  \bigg]\\
   &=\frac{-i}{2}\int\frac{d^3p}{(2\pi)^3}\bigg\{[\hat{a}_{1,-\vec{p}},\hat{a}_{2,\vec{p}}]+[\hat{a}_{2,-\vec{p}}^\dagger\hat{a}_{1,\vec{p}}^\dagger]+2(\hat{a}_{1,\vec p}\hat{a}_{2,\vec p}^\dagger-\hat{a}_{2,\vec p}\hat{a}_{1,\vec p}^\dagger)\bigg\}\\
   &=-i\int\frac{d^3p}{(2\pi)^3}(\hat{a}_{1,\vec p}\hat{a}_{2,\vec p}^\dagger-\hat{a}_{2,\vec p}\hat{a}_{1,\vec p}^\dagger)
\end{align*}
To prove this expression we used the fact that operators of different fields commutes.\\

We should observe that the simple conservation law leaves this quantity determined up to a multiplicative factor (which can be used to set the units of this charge) and a constant. This last constant can be removed using normal ordering:
\begin{equation*}
    \hat{\tilde{Q}}=(\hat Q +c)\quad\Rightarrow\quad \bra{0}\hat{\tilde{Q}}\ket{0}=c\quad\Rightarrow\quad :\hat{\tilde{Q}}:=\hat{\tilde{Q}}-\bra{0}\hat{\tilde{Q}}\ket{0}=\hat Q,
\end{equation*}
in this way the ambiguity that this constant could bring is removed.\\

Let's now find the spectrum of the charge operator we have just introduced. In order to do so we introduce the operators:
\begin{equation*}
    \hat{a}_{\pm}\triangleq\frac{1}{\sqrt{2}}(\hat{a}_{1,\vec p}\pm i\hat{a}_{2,\vec p}),\qquad\hat{a}^\dagger_{\pm}\triangleq\frac{1}{\sqrt{2}}(\hat{a}^\dagger_{1,\vec p}\mp i\hat{a^\dagger}_{2,\vec p}).
\end{equation*}
For these operators hold the following commutation relations:
\begin{align*}
    &[\hat Q,\hat{a}_{\pm,\vec q}]=-i\int\frac{d^3p}{(2\pi)^3}([\hat{a}_{1,\vec p}\hat{a}_{2,\vec p}^\dagger,\hat{a}_{\pm,\vec q}]-[\hat{a}_{2,\vec p}\hat{a}_{1,\vec p}^\dagger,\hat{a}_{\pm,\vec q}])\\
    &=-i\int\frac{d^3p}{(2\pi)^3}(\hat{a}_{1,\vec p}[\hat{a}_{2,\vec p}^\dagger,\hat{a}_{\pm,\vec q}]+[\hat{a}_{1,\vec p},\hat{a}_{\pm,\vec q}]\hat{a}_{2,\vec p}^\dagger-\hat{a}_{2,\vec p}[\hat{a}_{1,\vec p}^\dagger,\hat{a}_{\pm,\vec q}]-[\hat{a}_{2,\vec p},\hat{a}_{\pm,\vec q}]\hat{a}_{1,\vec p}^\dagger)\\
    &=-\frac{i}{\sqrt{2}}\int\frac{d^3p}{(2\pi)^3}(\mp i\hat{a}_{1,\vec p}(2\pi)^3\delta^3(\vec p-\vec q)+0+\hat{a}_{2,\vec p}(2\pi)^3\delta^3(\vec p-\vec q)-0)\\&=\frac{1}{\sqrt{2}}(\mp\hat{a}_{1,\vec q}-i\hat{a}_{2,\vec q}^\dagger)=\mp\hat{a}_{\pm},
\end{align*}
\begin{align*}
    &[\hat Q,\hat{a}_{\pm,\vec q}^\dagger]=-i\int\frac{d^3p}{(2\pi)^3}([\hat{a}_{1,\vec p}\hat{a}_{2,\vec p}^\dagger,\hat{a}_{\pm,\vec q}^\dagger]-[\hat{a}_{2,\vec p}\hat{a}_{1,\vec p}^\dagger,\hat{a}_{\pm,\vec q}^\dagger])\\
    &=-i\int\frac{d^3p}{(2\pi)^3}(\hat{a}_{1,\vec p}[\hat{a}_{2,\vec p}^\dagger,\hat{a}_{\pm,\vec q}^\dagger]+[\hat{a}_{1,\vec p},\hat{a}_{\pm,\vec q}^\dagger]\hat{a}_{2,\vec p}^\dagger-\hat{a}_{2,\vec p}[\hat{a}_{1,\vec p}^\dagger,\hat{a}_{\pm,\vec q}^\dagger]-[\hat{a}_{2,\vec p},\hat{a}_{\pm,\vec q}^\dagger]\hat{a}_{1,\vec p}^\dagger)\\
    &=-\frac{i}{\sqrt{2}}\int\frac{d^3p}{(2\pi)^3}(0+\hat{a}_{2,\vec p}^\dagger(2\pi)^3\delta^3(\vec p-\vec q)-0\pm i\hat{a}_{1,\vec p}^\dagger(2\pi)^3\delta^3(\vec p-\vec q))\\&=\frac{1}{\sqrt{2}}(\pm\hat{a}_{1,\vec q}-i\hat{a}_{2,\vec q}^\dagger)=\pm\hat{a}_{\pm}^\dagger.
\end{align*}
To prove these relations we have used the commutators:
\begin{equation*}
    [\hat{a}^\dagger_{i,\vec p},\hat{a}^\dagger_{j,\vec q}]=[\hat{a}_{i,\vec p},\hat{a}_{j,\vec q}]=0,\qquad[\hat{a}^\dagger_{i,\vec p},\hat{a}_{i,\vec q}]=(2\pi)^3\delta^3(\vec p-\vec q)\delta_{ij}.
\end{equation*}
Considering a state $\ket{s}$ with charge q ($\hat Q\ket s=q\ket s$), we can use the commutation relations above to prove that $\hat{a}_{\pm}^\dagger$ is a ladder operator of the charge (the same can be done for $\hat{a_{\pm}}$):
\begin{equation*}
    \hat{Q}\hat{a}^\dagger_{\pm} \ket{s}=([\hat{Q},\hat{a}^\dagger_{\pm}]+\hat{a}^\dagger_{\pm}\hat{Q})\ket{s}=(\pm\hat{a}^\dagger_{\pm}+\hat{a}^\dagger_{\pm}\hat{Q})\ket{s}=(q\pm1)\hat{a}^\dagger_{\pm}\ket{0}.
\end{equation*}
Since these ladder operators are linear combination of the ladder operators of the hamiltonian, 3-momentum and number operators, all the states generated by $\hat{a}^\dagger_{\pm}$ from vacuum are simultaneous eigenstates of $\hat{\mathcal{H}},\hat{\vec{p}},\hat{ N}$ and $\hat{ Q}$. This last one operator removes the degeneracy that was acquired considering two identical Klein-Gordon field, since there will be state with positive and others with negative charge (given the same momentum). \\
Therefore, every particle is fully described by its mass, its momentum and its charge. Opposite charge versions of the same particle will be interpreted as particle and antiparticle. Notice that this system can be reduced to a single real field only if the particles are all chargeless:
\begin{equation*}
    0=Q=\int d^3x(\dot\varphi_1\varphi_2-\dot\varphi_2\varphi_1)\quad \Rightarrow\quad \varphi_1=\varphi_2.
\end{equation*}

To end the discussion of charges and fields we can introduce a new single complex filed, built using the previous two:
\begin{equation*}
    \varphi=\frac{1}{\sqrt{2}}(\varphi_1+i\varphi_2),\qquad \varphi^*=\frac{1}{\sqrt{2}}(\varphi_1^*+i\varphi_2^*).
\end{equation*}
This field allows us to write the lagrangian density \eqref{KGLagrangeDensity2F} as:
\begin{equation}\label{KGLagrDensityComplex}
    \mathcal{L}=\partial_\mu\varphi^*\partial_\mu\varphi-m^2\varphi^*\varphi,
\end{equation}
this one is clearly invariant under $U(1)$ transformations, which is equivalent to the $O(2)$ symmetry of the system of two real fields we have just discussed. This symmetry leads again to a conserved current and charge:
\begin{align*}
    J^\mu&=\frac{\partial \mathcal{L} }{\partial\partial_\mu\varphi}\delta\varphi+\frac{\partial \mathcal{L} }{\partial\partial_\mu\varphi^*}\delta\varphi^*=(\partial^\mu\varphi^*)i\theta\varphi-(\partial^\mu\varphi)i\theta\varphi^*\\ &=i\theta(\varphi\partial^\mu\varphi^*-\varphi^*\partial^\mu\varphi),\\
    Q&=\int d^3x\ J^0=\int d^3x\ (\varphi\partial_t\varphi^*-\varphi^*\partial_t\varphi)=\int d^3x\ [\varphi\pi^*-\varphi^*\pi],\\
    \hat Q&=\int d^3x\ (\hat\varphi\hat\pi^*-\hat\varphi^*\hat\pi).
\end{align*}
Now, we can calculate these operators using the $\hat{a}_{\pm,\vec p}, \hat{a}^\dagger_{\pm,\vec p}$ operators, since they are defined as the same linear combination of creation/annihilation operators as the fields $\varphi,\ \varphi^*$:
\begin{equation*}
    \begin{cases*}
        \hat\varphi(\vec x)=\frac{1}{\sqrt{2}}\big(\hat\varphi_1(\vec x )+i\hat\varphi_2(\vec x )\big)=\int\frac{d^3p}{(2\pi)^3}\frac{1}{\sqrt{2\omega_{\vec{p}}}}\bigg(\hat{a}_{+,\vec p}e^{i\vec p\cdot\vec x}+\hat{a}^\dagger_{-,\vec p}e^{-i\vec p\cdot\vec x}\bigg),\\
        \hat\varphi^*(\vec x)=\frac{1}{\sqrt{2}}\big(\hat\varphi_1(\vec x )-i\hat\varphi_2(\vec x )\big)=\int\frac{d^3p}{(2\pi)^3}\frac{1}{\sqrt{2\omega_{\vec{p}}}}\bigg(\hat{a}_{-,\vec p}e^{i\vec p\cdot\vec x}+\hat{a}^\dagger_{+,\vec p}e^{-i\vec p\cdot\vec x}\bigg)
    \end{cases*}.
\end{equation*}
Using these (and the operator $\hat\pi$) we can recover all the operators we have already used, that in normal ordering read:
\begin{align*}
    \hat Q&=\int\frac{d^3p}{(2\pi)^3}(\hat{a}^\dagger_{+,\vec p}\hat{a}_{+,\vec p}+\hat{a}^\dagger_{-,\vec p}\hat{a}_{-,\vec p})=\hat N_+-\hat N_-,\\
    \hat N_{\pm}&=\int\frac{d^3p}{(2\pi)^3}\hat{a}^\dagger_{\pm,\vec p}\hat{a}_{\pm,\vec p}.
\end{align*}
The fields operators now represent particles/antiparticles couples, as $\hat{a}^\dagger_{+,\vec p}$ creates a particle and $\hat{a}^\dagger_{+,\vec p}$ an antiparticle. We should observe that these are always created with positive energies.
\section{Heisenberg picture}
Up until now, we have used the so-called \textbf{Schrödinger picture}, in which the time evolution is manged by a Schrödinger equation and a time evolution operator:
\begin{equation*}
    i\frac{d}{dt}\ket{\vec{p}(t)}=\hat{\mathcal{H} }\ket{\vec p(t)},\qquad \ket{\vec p(t)}=\hat{U}(0,t)\ket{\vec p(t)}=e^{-iE_{\vec{ p}}t}\ket{\vec p(0)}.
\end{equation*}
We will now introduce the \textbf{Heisenberg picture}, in which the time evolution is managed by every operator, in this way our field operators will be time-dependent:
\begin{equation*}
    \hat{O}_H(t)=e^{i\hat{\mathcal{H}} t}\hat{O}_Se^{-i\hat{\mathcal{H}} t}.
\end{equation*}
In this framework the time evolution of every operator is determined by its commutator with the hamiltonian at a time fixed:
\begin{align*}
    \frac{d}{dt}\hat{O}(t)&=\frac{de^{i\hat{\mathcal{H}} t}}{dt}\hat{O}_Se^{-i\hat{\mathcal{H}} t}+e^{i\hat{\mathcal{H}} t}\hat{O}_S\frac{de^{-i\hat{\mathcal{H}} t}}{dt}\\&=i\hat{\mathcal{H} }\hat{O}_H(t)-i\hat{O}_H(t)\hat{\mathcal{H} }=i[\hat{\mathcal{H} },\hat{O}_H(t)].
\end{align*}
Recalling the commutators in Schrödinger picture
\begin{equation*}
    \begin{cases}
        [\hat{\varphi}(\vec x),\hat{\varphi}(\vec y)]=[\hat{\pi}(\vec x), \hat{\pi}(\vec y)]=0\\
        [\hat{\varphi}(\vec x), \hat{\pi}(\vec y)]=i\delta^3(\vec x-\vec y)
    \end{cases}
\end{equation*}
we can get the time evolution equation in the Heisenberg picture ($\varphi(t,\vec x)=\varphi(x)$)
\begin{align*}
    \frac{\partial\hat{\varphi}(x)}{\partial t}&=i[\hat{\mathcal{H} },\hat{\varphi}(x)]=\frac{i}{2}\bigg[\int d^3y(\hat{\pi}^2+\big(\vec\nabla\hat{\varphi})^2+m^2\hat\varphi^2\big),\hat\varphi\bigg]_{\text{fixed t}}\\&=\frac{i}{2}\int d^3y(\hat\pi[\hat\pi,\hat\varphi]+[\hat\pi,\hat\varphi]\hat\pi)=\frac{1}{2}\int d^3y(\hat\pi\delta^3(\vec x-\vec y)+\delta^3(\vec x-\vec y)\hat\pi)\\&=\hat\pi(x),
    \end{align*}\begin{align*}
    \frac{\partial\hat{\pi}(x)}{\partial t}&=i[\hat{\mathcal{H} },\hat{\pi}(x)]=\frac{i}{2}\bigg[\int d^3y(\hat{\pi}^2+\big(\vec\nabla_y\hat{\varphi})^2+m^2\hat\varphi^2\big),\hat\pi\bigg]_{\text{fixed t}}\\
    &=\frac{i}{2}\int d^3y([(\vec\nabla_y\hat\varphi)^2,\hat\pi]+m^2[\hat\varphi^2,\hat\pi])\\&=\frac{i}{2}\int d^3y(\vec\nabla_y\hat\varphi\cdot[\vec\nabla_y\hat\varphi,\hat\pi]+[\vec\nabla_y\hat\varphi,\hat\pi]\cdot\vec\nabla_y\hat\varphi+m^2[\hat\varphi^2,\hat\pi])\\
    &=-\int d^3y(\vec\nabla_y\hat\varphi\cdot\vec\nabla_y\delta^3(\vec x-\vec y)+m^2\hat\varphi\delta^3(\vec x-\vec y))\\&\qquad\text{ Integrating by parts}\\&=\nabla^2_x\hat\varphi(x)-m^2\hat\varphi(x).
\end{align*}
Combining these two we can get a differential equation for $\hat\varphi$:
\begin{equation*}
    \ddot{\hat{\varphi}}=(\vec\nabla^2-m^2)\hat\varphi,
\end{equation*}
which we can easily recognize as the Klein-Gordon equation \eqref{KleinGordonEq}, confirming that our second quantization procedure is consistent with the starting point of this theory.\\

We can now study the time evolution of the creation/annihilation operators:
\begin{equation*}
    (\hat{a}_{\vec p})_H=e^{i\hat{\mathcal{H}} t}\hat{a}_{\vec p}e^{-i\hat{\mathcal{H}} t}=\big([e^{i\hat{\mathcal{H}} t},\hat{a}_{\vec p}]+\hat{a}_{\vec p}e^{i\hat{\mathcal{H}} t}\big)e^{-i\hat{\mathcal{H}} t},
\end{equation*}
in order to proceed in this calculation we should evaluate the above commutator, which, using the Taylor series definition of the exponential of an operator, reduces to the following calculations
\begin{align*}
   [\hat{\mathcal{H}},\hat{a}_{\vec p}]&=\int\frac{d^3q}{(2\pi)^3}\omega_{\vec{q}}[\hat{a}^\dagger_{\vec{q}}\hat{a}_{\vec{q}},\hat{a}_{\vec p}]=\int\frac{d^3q}{(2\pi)^3}\omega_{\vec{q}}(\hat{a}^\dagger_{\vec{q}}[\hat{a}_{\vec{q}},\hat{a}_{\vec p}]+[\hat{a}^\dagger_{\vec{q}},\hat{a}_{\vec p}]\hat{a}_{\vec{q}})\\ &=-\int\frac{d^3q}{(2\pi)^3}\omega_{\vec{q}}(2\pi)^3\delta^3(\vec p-\vec q)\hat{a}_{\vec{q}}=-\omega_{\vec p}\hat{a}_{\vec p},\\
   \Rightarrow\qquad\ \ \hat{\mathcal{H} }\hat{a}_{\vec p}&=\hat{a}_{\vec p}(\hat{\mathcal{H} }-E_{\vec p}),\\
   \Rightarrow\qquad\ \hat{\mathcal{H} }^n\hat{a}_{\vec p}&=\hat{a}_{\vec p}(\hat{\mathcal{H} }-E_{\vec p})^{n},\\\Rightarrow\quad [e^{i\hat{\mathcal{H}} t},\hat{a}_{\vec p}]&=e^{i\hat{\mathcal{H} }t}\hat{a}_{\vec p}-\hat{a}_{\vec p}e^{i\hat{\mathcal{H} }t}=\hat{a}_{\vec p}e^{i(\hat{\mathcal{H} }-E_{\vec{p}})t}-\hat{a}_{\vec p}e^{i\hat{\mathcal{H} }t}=(e^{-E_{\vec p}t}-1)\hat{a}_{\vec p}e^{i\hat{\mathcal{H} }t},\\
   \Longrightarrow &\qquad(\hat{a}_{\vec p})_H =e^{-iE_{\vec{p}}t}\hat{a}_{\vec{p}},\qquad(\hat{a}^\dagger_{\vec p})_H=e^{-E_{\vec{p}}t}\hat{a}^\dagger_{\vec{p}}.
\end{align*}
Therefore, the field operator now reads:
\begin{equation*}
    \varphi(t,\vec x)=\int\frac{d^3p}{(2\pi)^3}\frac{1}{\sqrt{2\omega_{\vec p}}}\bigg(\hat{a}_{\vec p}e^{-i(E_{\vec p}t-\vec p\cdot\vec x)}+\hat{a}^\dagger_{\vec p}e^{+i(E_{\vec p}t-\vec p\cdot\vec x)}\bigg)=\int\frac{d^3p}{(2\pi)^3}\frac{1}{\sqrt{2\omega_{\vec p}}}\bigg(\hat{a}_{\vec p}e^{-ipx}+\hat{a}^\dagger_{\vec p}e^{+ipx}\bigg),
\end{equation*}
where $px$ stands for $P^\mu x_\mu$.\\We shall observe that in this way we obtained a Fourier expansion in time independent operators and in which time dependency is managed by the exponential terms.\\

Now that we have introduced time dependency, we should check if everything in our theory is Lorentz consistent. We have already checked the Lorentz invariance of this framework, we still have to check if it holds causality.\\First of all, let's notice that two operators should always commute if they represent observables measured in two space-time events which are reciprocally outside their light cones. In fact, in these cases the two measures cannot be dependent on each other (in the sense that there cannot be any correlation due to causality). 
\begin{equation}
    [\hat O_1(x),\hat O_2(y)]=0\quad \forall x,t\ \text{such that } (x-y)^2=(x^\mu-y^\mu)(x_\mu-y_\mu)<0. 
\end{equation} 
In order to check this requirement we define the propagator
\begin{align}
    \label{KGpropagator}\Delta(x-y)&=[\hat\varphi(x),\hat\varphi(y)]=\int\frac{d^3p}{(2\pi)^3}\frac{d^3q}{(2\pi)^3}\frac{(2\pi)^3}{2\sqrt{E_{\vec p}E_{\vec q}}}\delta^3(\vec p-\vec q)(e^{-ipx+iqy}-e^{ipx-iqy})\nonumber\\&=\int\frac{d^3p}{(2\pi)^3}\frac{1}{2E_{\vec p}}(e^{-ip(x-y)}-e^{ip(x-y)}).
\end{align}
We should notice that this operator is Lorentz invariant, due to the invariance of the measure of the integral and of the argument of the exponential.\\

Let's study the \eqref{KGpropagator} inside the light-cone, therefore we will consider an inertial reference frame in which the events $x,\ y$ happens at the same point in space, furthermore we will require $x-y$ to be a space-like 4-vector:
\begin{equation*}
    (x-y)^2=t^2>0.
\end{equation*}
In this case the propagator \eqref{KGpropagator} reads:
\begin{equation*}
    \Delta(x-y)=\frac{4\pi}{(2\pi)^3}\int_0^{\infty}\ dp\frac{\vec p^2}{2\sqrt{\vec p^2+m^2}}(e^{-i\sqrt{\vec p^2+m^2}t}-e^{i\sqrt{\vec p^2+m^2}t}),
\end{equation*}
this integral results in a combination of the Bessel's functions $J_1(mt)$ and $Y_1(mt)$, in the limit of $t\rightarrow\infty$ those tend to 
\begin{equation*}
    \Delta(x-y)\sim (e^{-imt}-e^{imt})\neq0.
\end{equation*}
In this way we see that event inside the light-cone can be non-vanishing.\\

Considering $x-y$ time-like, in a reference frame where the two events happens at the same time, such that:
\begin{equation*}
    (x-y)^2=-r^2<0.
\end{equation*}
Now, in this case the \eqref{KGpropagator} reads:
\begin{align*}
    \Delta(x-y)&=\int\frac{d^3p}{(2\pi)^3}\frac{1}{2\sqrt{\vec p^2+m^2}}(e^{-i\vec p\cdot\vec r}-e^{i\vec p\cdot\vec r})\\&=\int\frac{d^3p}{(2\pi)^3}\frac{1}{2\sqrt{\vec p^2+m^2}}e^{-i\vec p\cdot\vec r}-\int\frac{d^3p}{(2\pi)^3}\frac{1}{2\sqrt{\vec p^2+m^2}}e^{i\vec p\cdot\vec r}
    \\&=\int\frac{d^3p}{(2\pi)^3}\frac{1}{2\sqrt{\vec p^2+m^2}}e^{-i\vec p\cdot\vec r}-\int\frac{d^3p}{(2\pi)^3}\frac{1}{2\sqrt{\vec p^2+m^2}}e^{-i\vec p\cdot\vec r}=0,
\end{align*}
in which we have changed variables in the second integral $\vec p\rightarrow-\vec p$. This shows that the propagator always vanishes outside the light-cone granting that causality holds (since observables depending on the field outside and inside the light-cone cannot be correlated).\\

Let's now try to compute the probability of a particle to exit the light cone, this will be evaluated as the modulus squared of
\begin{align*}
    D(x-y)&=\bra{0}\hat\varphi^\dagger(x)\hat\varphi(y)\ket{0}\\&=\int\frac{d^3p\ d^3q}{(2\pi)^6}\frac{1}{2\sqrt{E_{\vec p}E_{\vec q}}}\bra{0}(\hat{a}_{\vec p}e^{-ipx}+\hat{a}^\dagger_{\vec p}e^{ipx})(\hat{a}_{\vec q}e^{-iqy}+\hat{a}^\dagger_{\vec q}e^{iqy})\ket{0}\\&=\int\frac{d^3p\ d^3q}{(2\pi)^6}\frac{1}{2\sqrt{E_{\vec p}E_{\vec q}}}\bra{0}\hat{a}_{\vec p}\hat{a}^\dagger_{\vec q}\ket{0}e^{-ipx+iqy}\\&=\int\frac{d^3p\ d^3q}{(2\pi)^6}\frac{1}{2\sqrt{E_{\vec p}E_{\vec q}}}\bra{0}([\hat{a}_{\vec p},\hat{a}^\dagger_{\vec q}]+\hat{a}^\dagger_{\vec q}\hat{a}_{\vec p})\ket{0}e^{-ipx+iqy}\\&=\int\frac{d^3p\ d^3q}{(2\pi)^6}\frac{e^{-ipx+iqy}}{2\sqrt{E_{\vec p}E_{\vec q}}}\bra{0}([\hat{a}_{\vec p},\hat{a}^\dagger_{\vec q}])\ket{0}=\int\frac{d^3p\ d^3q}{(2\pi)^3}\frac{e^{-ipx+iqy}}{2\sqrt{E_{\vec p}E_{\vec q}}}\delta^3(\vec p-\vec q)\\&=\int\frac{d^3p}{(2\pi)^3}\frac{1}{2E_{\vec p}}e^{-ip(x-y)},
\end{align*}
imposing that $(x-y)^2<0$.\\
To simplify this calculation we can evaluate this integral in a reference frame where the two events happens at the same time
\begin{align*}
    D(x-y)&=\int\frac{d^3p}{(2\pi)^3}\frac{1}{2E_{\vec p}}e^{i\vec p\cdot\vec r}=\frac{1}{(2\pi)^3}\int_{0}^{2\pi}d\varphi\int_{-1}^{1}d(cos\theta)e^{i|\vec p|r cos\theta}\int_{0}^{\infty}\frac{|\vec p|^2dp}{2E_{\vec p}}\\&=\frac{2\pi}{(2\pi)^3}\int_{0}^{\infty}dp\frac{e^{ipr}-e^{-ipr}}{2irE_{\vec p}}p=\frac{-i}{2(2\pi)^2r}\int_{-\infty}^{+\infty}dp\frac{p}{\sqrt{p^2+m^2}}e^{ipr}.
\end{align*}
To proceed we need to integrate over the complex plane, in the same way as we did in the introduction, obtaining a non-vanishing integral:
\begin{equation*}
    D(x-y)=-\frac{2i}{2(2\pi)^2r}\int_{m}^{+\infty}dp\frac{-ye^{iyr}}{i\sqrt{y^2-m^2}}=\frac{1}{(2\pi)^2r}\int_{m}^{+\infty}dy\frac{ye^{iyr}}{i\sqrt{y^2-m^2}}\neq0.
\end{equation*}
This seems to contradict what we previously proved about causality, since there is a non-zero probability of a particle to exit the light-cone.\\What really happens is that a particle can actually exit the light-cone but it cannot generate correlations, since we have proved that $\Delta(x-y)$ would vanish.\\Now, we could physically interpret this strange behavior as the interference of another particle entering the light-cone, since it holds:
\begin{equation*}
    \bra{0}\Delta(x-y)\ket{0}=\bra{0}\hat\varphi^\dagger(x)\hat\varphi(y)\ket{0}-\bra{0}\hat\varphi(y)\hat\varphi^\dagger(x)\ket{0}=D(x-y)-D(y-x)=0.
\end{equation*}
Considering a complex Klein-Gordon field we would get that the particle entering the light cone is actually the antiparticle of the field.

